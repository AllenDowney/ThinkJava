% LaTeX source for ``Think Java: How to think like a computer scientist''
% Copyright (C) 2012  Allen B. Downey

% Permission is granted to copy, distribute, transmit and adapt
% this work under a Creative Commons
% Attribution-NonCommercial-ShareAlike 3.0 Unported License:
% http://creativecommons.org/licenses/by-nc-sa/3.0/

% If you are interested in distributing a commercial version of this
% work, please contact Allen B. Downey.

% The original form of this book is LaTeX source code.  Compiling this
% LaTeX source has the effect of generating a device-independent
% representation of the book, which can be converted to other formats
% and printed.

% The LaTeX source for this book is available from http://thinkapjava.com
% and http://code.google.com/p/thinkapjava/source/checkout

% This book was typeset using LaTeX .  The illustrations were
% drawn in xfig.  All of these are free, open-source programs.

%%----------------------------------------------------------------

% How to compile this document:
% If your environment provides latex, makeindex, and dvips,
% the following commands should produce a Postscript version
% of the book.

%        latex book
%        makeindex book
%        latex book
%        dvips -o book.ps book

% You will also need the following (fairly standard) latex
% packages: url, epsfig, makeidx, fancyhdr

% This distribution also includes a Makefile that should
% compile both the Postscript and PDF versions of the book.

%%-----------------------------------------------------------------

\documentclass[12pt]{book}

\usepackage{amsmath}
\usepackage{amsthm}
\usepackage{amssymb}
\usepackage{makeidx}
\usepackage{url}
\usepackage{fancyhdr}
\usepackage{verbatim}
\usepackage{moreverb}
\usepackage{hevea}
\usepackage{upquote}
\usepackage{graphicx}


\newcommand{\thetitle}{Think Java: How to Think Like a Computer Scientist}
\newcommand{\theversion}{5.1.2}

\newif\ifplastex
\plastexfalse

\ifplastex
% PLASTEX ONLY
    \usepackage{localdef}
    \maketitle

\else

% LATEX AND HTML

\newtheoremstyle{exercise}
  {\topsep}     % space above
  {\topsep}     % space below
  {}            % body font
  {}            % indent amount
  {\bfseries}   % head font
  {.}           % punctuation
  {5pt}         % head space
  {}            % custom head
\theoremstyle{exercise}
\newtheorem{exercise}{Exercise}[chapter]

\title{\thetitle}

\author{Allen B. Downey}

% these styles get translated in CSS for the HTML version
\newstyle{a:link}{color:black;}
\newstyle{pre}{color:darkgreen;font-size:110\%;}
\newstyle{tt}{color:darkgreen;font-size:110\%;}
\newstyle{p+p}{margin-top:1em;margin-bottom:1em}
\newstyle{img}{border:0px}

% change the arrows
\setlinkstext
  {\imgsrc[ALT="Previous"]{back.png}}
  {\imgsrc[ALT="Up"]{up.png}}
  {\imgsrc[ALT="Next"]{next.png}}

\makeindex

% define styles for syntax highlighting in code listings
\usepackage[usenames,dvipsnames]{color}
\usepackage{listings}
\lstset{
    language=java,
    basicstyle=\ttfamily,
    commentstyle=\color{Green},
    keywordstyle=\color{blue},
    stringstyle=\color{Orange},
    columns=fullflexible,
    keepspaces=true,
    showstringspaces=false,
    aboveskip=4pt,
    belowskip=-4pt
}

% enable pdf hyperlinks, table of contents, and document properties
\usepackage[pdftex]{hyperref}
\makeatletter
\hypersetup{%
  pdftitle={\@title},
  pdfauthor={\@author},
  pdfsubject={Version \theversion},
  pdfkeywords={},
  bookmarksopen=false,
  colorlinks=true,
  citecolor=black,
  filecolor=black,
  linkcolor=black,
  urlcolor=blue
}
\makeatother

\begin{document}

\frontmatter

% title page for the HTML version

\begin{htmlonly}

% TITLE PAGE FOR HTML VERSION

{\Large Think Java}

{\large Allen B. Downey}

Version \theversion

\setcounter{chapter}{-1}

\end{htmlonly}

% BEGINNING OF LATEXONLY

\input{latexonly}

\begin{latexonly}

%-half title--------------------------------------------------

%\thispagestyle{empty}
%
%\begin{flushright}
%\vspace*{2.5in}
%
%{\huge Think Java}
%
%\vspace{1in}
%
%{\LARGE How to Think Like a Computer Scientist}
%
%\vfill
%
%\end{flushright}

%--verso------------------------------------------------------

%%\clearemptydoublepage
%\cleardoublepage

%--title page--------------------------------------------------
\pagebreak
\thispagestyle{empty}

\begin{flushright}
\vspace*{2.5in}

{\huge Think Java}

\vspace{0.25in}

{\LARGE How to Think Like a Computer Scientist}

\vspace{1in}

{\Large
Allen B. Downey
}


\vspace{1in}

{\Large \theversion}

\vfill

\end{flushright}


%--copyright--------------------------------------------------
\pagebreak
\thispagestyle{empty}

Copyright \copyright ~2012 Allen Downey.

\vspace{0.25in}

Permission is granted to copy, distribute, transmit and adapt
this work under a Creative Commons
Attribution-NonCommercial-ShareAlike 3.0 Unported License:
\url{http://creativecommons.org/licenses/by-nc-sa/3.0/}

If you are interested in distributing a commercial version of this
work, please contact Allen B. Downey.

The original form of this book is \LaTeX\ source code.  Compiling this
\LaTeX\ source has the effect of generating a device-independent
representation of the book, which can be converted to other formats
and printed.

The \LaTeX\ source for this book is available from:
\url{http://thinkapjava.com}

This book was typeset using \LaTeX .  The illustrations were
drawn in xfig.  All of these are free, open-source programs.

\vspace{0.25in}


%-----------------------------------------------------------------

\end{latexonly}

\fi
% END OF THE PART WE SKIP FOR PLASTEX

