\chapter{Program development}
\label{development}

\section{Strategies}

I present different program development strategies throughout
the book, so I wanted to pull them together here.
%
The foundation of all strategies is {\bf incremental development},
which goes like this:

\begin{enumerate}

\item Start with a working program that does something visible,
   like printing something.

\item Add a small number of lines of code at a time,
   and test the program after every change.

\item Repeat until the program does what it is supposed to do.

\end{enumerate}

After every change, the program should produce some visible effect
that tests the new code.  This approach to programming can save
a lot of time.

Because you only add a few lines of code at a time, it
is easy to find syntax errors.
%
And because each version of the
program produces a visible result, you are constantly testing your
mental model of how the program works.  If your mental model is
wrong, you are confronted with the conflict (and have a chance
to correct it) before you write a lot of bad code.

The challenge of incremental development is that is it not
easy to figure out a path from the starting place
to a complete and correct program.
%
To help with that, there are several strategies to choose from:

\begin{description}

\item[Encapsulation and generalization:] If you don't know yet how
to divide the computation into methods, start writing code in
{\tt main}, then look for coherent chunks to encapsulate in
a method, and generalize them appropriately.

\item[Rapid prototyping:] If you know what method to write, but not
how to write it, start with a rough draft that handles the simplest
case, then test it with other cases, extending and correcting as you go.

\item[Bottom-up:] Start by writing simple methods, then assemble them
into a solution.

\item[Top-down:] Use pseudocode to design the structure of the
computation and identify the methods you'll need.  Then write the
methods and replace the pseudocode with real code.

\end{description}

Along the way, you might need some scaffolding. For example, each
class should have a {\tt toString} method that lets you print the
state of an object in human-readable form.  This method is useful
for debugging, but usually not part of a finished program.

\section{Failure modes}

If you are spending a lot of time debugging, it is probably
because you are using an ineffective development strategy.  Here
are the failure modes I see most often (and occasionally fall into):

\begin{description}

\item[Non-incremental developement:] If you write more than a few
  lines of code without compiling and testing, you are asking for
  trouble.  One time when I asked a student how the homework was
  coming along, he said, ``Great!  I have it all written.  Now I just
  have to debug it.''

\item[Attachment to bad code:] If you write more than a few lines of
  code without compiling and testing, you may not be able to debug it.
  Ever.  Sometimes the only strategy is (gasp!)~to delete the bad
  code and start over (using an incremental strategy).  But beginners
  are often emotionally attached to their code, even if it doesn't
  work.  The only way out of this trap is to be ruthless.

\item[Random-walk programming:] I sometimes work with students who
  seem to be programming at random.  They make a change, run the
  program, get an error, make a change, run the program, etc.  The
  problem is that there is no apparent connection between the outcome
  of the program and the change.
%
  If you get an error message, take the time to read it.
  More generally, take time to think.

\item[Compiler submission:] Error messages are useful, but they
  are not always right.  For example, if the message says, ``Semi-colon
  expected on line 13,'' that means there is a syntax error near
  line 13. But putting a semi-colon on line 13 is not always the
  solution.  Don't submit to the will of the compiler.

\end{description}

The next chapter makes more suggestions for effective debugging.


