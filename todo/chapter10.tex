\chapter{GridWorld: Part 2}
\label{gridworld2}

Part 2 of the GridWorld case study uses some features we haven't
seen yet, so you will get a preview now and more details later.
As a reminder, you can find the
documentation for the GridWorld classes at
\url{http://www.greenteapress.com/thinkapjava/javadoc/gridworld/}.

When you install GridWorld, you should have a folder named {\tt
  projects/boxBug}, which contains {\tt BoxBug.java}, {\tt
  BoxBugRunner.java} and {\tt BoxBug.gif}.

Copy these files into your working folder and import them into
your development environment.
There are instructions here that
might help: \url{http://www.collegeboard.com/prod_downloads/student/testing/ap/compsci_a/ap07_gridworld_installation_guide.pdf}.

Here is the code from {\tt BoxBugRunner.java}:

\begin{lstlisting}
import info.gridworld.actor.ActorWorld;
import info.gridworld.grid.Location;

import java.awt.Color;

public class BoxBugRunner {
    public static void main(String[] args) {
        ActorWorld world = new ActorWorld();
        BoxBug alice = new BoxBug(6);
        alice.setColor(Color.ORANGE);
        BoxBug bob = new BoxBug(3);
        world.add(new Location(7, 8), alice);
        world.add(new Location(5, 5), bob);
        world.show();
    }
}
\end{lstlisting}

Everything here should be familiar, with the possible exception of
{\tt Location}, which is part of GridWorld, and similar to {\tt
  java.awt.Point}.

{\tt BoxBug.java} contains the class definition for BoxBug.

\begin{lstlisting}
public class BoxBug extends Bug {
    private int steps;
    private int sideLength;

    public BoxBug(int length) {
        steps = 0;
        sideLength = length;
    }
}
\end{lstlisting}

The first line says that this class extends {\tt Bug}, which means
that a {\tt BoxBug} is a kind of {\tt Bug}.

The next two lines are instance variables.  Every {\tt Bug}
has variables named {\tt sideLength}, which determines the size of the
box it draws, and {\tt steps}, which keeps track of how many steps
the {\tt Bug} has taken.

The next line defines a {\bf constructor}, which is a special
method that initializes the instance variables.  When you create
a {\tt Bug} by invoking {\tt new}, Java invokes this constructor.

The parameter of the constructor is the side length.

{\tt Bug} behavior is controlled by the {\tt act} method.  Here
is the {\tt act} method for {\tt BoxBug}:

\begin{lstlisting}
    public void act() {
        if (steps < sideLength && canMove()) {
            move();
            steps++;
        }
        else {
            turn();
            turn();
            steps = 0;
        }
    }
\end{lstlisting}

If the {\tt BoxBug} can move, and has not taken the required number
of steps, it moves and increments {\tt steps}.

If it hits a wall or completes one side of the box, it turns 90 degrees
to the right and resets {\tt steps} to 0.

Run the program and see what it does.  Did you get the behavior you
expected?


\section{Termites}

I wrote a class called {\tt Termite} that extends {\tt Bug} and adds
the ability to interact with flowers. To run it, download these files
and import them into your development environment:

\url{http://thinkapjava.com/code/Termite.java} \\
\url{http://thinkapjava.com/code/Termite.gif} \\
\url{http://thinkapjava.com/code/TermiteRunner.java} \\
\url{http://thinkapjava.com/code/EternalFlower.java}

Because {\tt Termite} extends {\tt Bug}, all {\tt Bug} methods
also work on {\tt Termite}s.  But {\tt Termite}s have additional
methods that {\tt Bug}s don't have.

\begin{lstlisting}
    /**
     * Returns true if the termite has a flower.
     */
    public boolean hasFlower();

    /**
     * Returns true if the termite is facing a flower.
     */
    public boolean seeFlower();

    /**
     * Creates a flower unless the termite already has one.
     */
    public void createFlower();

    /**
     * Drops the flower in the termites current location.
     *
     * Note: only one Actor can occupy a grid cell, so the effect
     * of dropping a flower is delayed until the termite moves.
     */
    public void dropFlower();

    /**
     * Throws the flower into the location the termite is facing.
     */
    public void throwFlower();

    /**
     * Picks up the flower the termite is facing, if there is
     * one, and if the termite doesn't already have a flower.
     */
    public void pickUpFlower();
\end{lstlisting}

For some methods {\tt Bug} provides one definition and {\tt Termite}
provides another.  In that case, the {\tt Termite} method
{\bf overrides} the {\tt Bug} method.

For example, {\tt Bug.canMove}
returns {\tt true} if there is a flower in the next location, so
{\tt Bug}s can trample {\tt Flower}s.  {\tt Termite.canMove} returns
{\tt false} if there is any object in the next location, so
{\tt Termite} behavior is different.

As another example, Termites have a version of {\tt turn} that takes
an integer number of degrees as a parameter.  Finally, Termites
have {\tt randomTurn}, which turns left or right 45 degrees at random.

% TODO: move for loops into the string chapter.

Here is the code from {\tt TermiteRunner.java}:

\begin{lstlisting}
public class TermiteRunner
{
    public static void main(String[] args)
    {
        ActorWorld world = new ActorWorld();
        makeFlowers(world, 20);

        Termite alice = new Termite();
        world.add(alice);

        Termite bob = new Termite();
        bob.setColor(Color.blue);
        world.add(bob);

        world.show();
    }

    public static void makeFlowers(ActorWorld world, int n) {
        for (int i = 0; i<n; i++) {
            world.add(new EternalFlower());
        }
    }
}
\end{lstlisting}

Everything here should be familiar.  {\tt TermiteRunner} creates
an {\tt ActorWorld} with 20 {\tt EternalFlowers} and two {\tt Termites}.

An {\tt EternalFlower} is a {\tt Flower} that overrides {\tt act}
so the flowers don't get darker.

\begin{lstlisting}
public class EternalFlower extends Flower {
    public void act() {}
}
\end{lstlisting}

If you run {\tt TermiteRunner.java} you should see two termites
moving at random among the flowers.

{\tt MyTermite.java} demonstrates the methods that interact with
flowers.  Here is the class definition:

\begin{lstlisting}
public class MyTermite extends Termite {

    public void act() {
        if (getGrid() == null)
            return;

        if (seeFlower()) {
            pickUpFlower();
        }
        if (hasFlower()) {
            dropFlower();
        }

        if (canMove()) {
            move();
        }
        randomTurn();
    }
}
\end{lstlisting}

{\tt MyTermite} extends {\tt Termite} and overrides {\tt act}.
If {\tt MyTermite} sees a flower, it picks it up.  If it has
a flower, it drops it.


\section{Langton's Termite}

Langton's Ant is a simple model of ant behavior that
displays surprisingly complex behavior.  The Ant lives on a grid
like GridWorld where each cell is either white or black.  The
Ant moves according to these rules:

\begin{itemize}

\item If the Ant is on a white cell, it turns to the right,
makes the cell black, and moves forward.

\item If the Ant is on a black cell, it turns to the left,
makes the cell white, and moves forward.

\end{itemize}

Because the rules are simple, you might expect the Ant to do
something simple like make a square or repeat a simple
pattern.  But starting on a grid with all white cells, the
Ant makes more than 10,000 steps in a seemingly random pattern
before it settles into a repeating loop of 104 steps.

You can read more about Langton's
Ant at \url{http://en.wikipedia.org/wiki/Langton_ant}.

It is not easy to implement
Langton's Ant in GridWorld because we can't set the color of
the cells.  As an alternative, we can use Flowers to mark
cells, but we can't have an Ant and a Flower in the same cell,
so we can't implement the Ant rules exactly.

Instead we'll create what I'll call a {\tt LangtonTermite}, which uses
{\tt seeFlower} to check whether there is a flower in the next cell,
{\tt pickUpFlower} to pick it up,
and {\tt throwFlower} to put a flower in the next cell.  You
might want to read the code for these methods to be sure you
know what they do.

\section{Exercises}

\begin{exercise}
Now you know enough to do the exercises in the Student Manual, Part 2.
Go do them, and then come back for more fun.
\end{exercise}

\begin{exercise}
The purpose of this exercise is to explore the behavior of Termites
that interact with flowers.

Modify {\tt TermiteRunner.java} to create {\tt MyTermites} instead
of {\tt Termites}.  Then run it again.  {\tt MyTermites} should move
around at random, moving the flowers around.  The total number of
flowers should stay the same (including the ones {\tt MyTermites}
are holding).

In {\em Termites, Turtles and Traffic Jams}, Mitchell Resnick describes
a simple model of termite behavior:

\begin{itemize}

\item If you see a flower, pick it up.  Unless you already have
a flower; in that case, drop the one you have.

\item Move forward, if you can.

\item Turn left or right at random.

\end{itemize}

Modify {\tt MyTermite.java} to implement this model.  What effect do
you think this change will have on the behavior of {\tt MyTermites}?

Try it out.  Again, the total number of flowers does not change,
but over time the flowers accumulate in a small number of piles, often
just one.

This behavior is an {\bf an emergent property}, which you can read
about at \url{http://en.wikipedia.org/wiki/Emergence}.  {\tt MyTermites}
follow simple rules using only small-scale information, but the result
is large-scale organization.

Experiment with different rules and see what effect they have on the
system.  Small changes can have unpredicable results!

\end{exercise}

\begin{exercise}

\begin{enumerate}

\item Make a copy of {\tt Termite.java} called {\tt LangtonTermite}
and a copy of {\tt TermiteRunner.java} called {\tt LangtonRunner.java}.
Modify them so the class definitions have the same name as the files,
and so {\tt LangtonRunner} creates a {\tt LangtonTermite}.

\item If you create a file named {\tt LangtonTermite.gif}, GridWorld
  uses it to represent your Termite.  You can download excellent
  pictures of insects from
  \url{http://www.cksinfo.com/animals/insects/realisticdrawings/index.html}.
  To convert them to GIF format, you can use an application like
  ImageMagick.

\item Modify {\tt act} to implement rules similar to Langton's Ant.
Experiment with different rules, and with both 45 and 90 degree turns.
Find rules that run the maximum number of steps before the Termite
starts to loop.

\item To give your Termite enough room, you can make the grid
bigger or switch to an {\tt UnboundedGrid}.

\item Create more than one {\tt LangtonTermite} and see how they
interact.

\end{enumerate}

\end{exercise}


