\section{Vocabulary}

\begin{description}

\term{syntax error}
An error in a program that makes it impossible to parse (and therefore impossible to compile).

%\term{type-safe}
%A property of Java that makes it possible to catch some errors at compile time.

\term{exception}
An error in a program that makes it impossible to execute completely.
Also called a runtime error.

\term{logic error}
An error in a program that makes it do something other than what the programmer intended.

\term{variable}
A named storage location for values.
All variables have a type, which is declared when the variable is created.

\term{value}
A number or string that can be stored in a variable.
Every value belongs to a type (for example, \java{int} or \java{String}).

\term{declaration}
A statement that creates a new variable and specifies its type.

\term{type}
Mathematically speaking, a set of values.
The type of a variable determines which values it can have.

\term{assignment}
A statement that stores a value in a memory location.

%\term{literal}
%A constant value written directly in the source code.
%For example, \java{"Hello"} is a string literal and \java{74} is an integer literal.

%\term{constant}
%A variable that can only be assigned one time.
%Once initialized, its value cannot be changed.

\term{initialize}
To assign an initial value to a variable.

\term{keyword}
A reserved word used by the compiler to parse programs.
You cannot use keywords (like \java{public}, \java{class}, and \java{void}) as variable names.

\term{expression} 
A combination of variables, operators, and values that represents a single value.  Expressions also have types, as determined by their operators and operands.

\term{operator}
A symbol that represents a computation like addition, multiplication, or string concatenation.

%\term{operand}
%One of the values on which an operator operates.

\term{floating-point}
A data type that represents decimal numbers (numbers that have an integer
part and a fractional part).
In Java, the default floating-point type is \java{double}.

\term{concatenate}
To join two operands end-to-end.
For strings, concatenation means to append.

\term{precedence}
The order in which operations are evaluated.

\term{composition}
The ability to combine simple expressions and statements into compound expressions and statements, making it possible to represent complex computations in a concise manner.

\term{whitespace}
Newlines, tab characters, and other spaces in a source program.
Its purpose in the Java language is to separate tokens.

%\term{wildcard}
%A command-line feature that allows you to specify a pattern of file names.

\end{description}
