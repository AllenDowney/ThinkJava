\section{Glossary}

\begin{description}

\item[variable:] A named storage location for values.  All
variables have a type, which is declared when the variable
is created.

\item[value:] A number or string (or other thing to be named later)
that can be stored in a variable.  Every value belongs to a type.

\item[type:] A set of values.  The type of a variable
determines which values can be stored there.  The types
we have seen are integers ({\tt int} in Java) and strings
({\tt String} in Java).

\item[keyword:]  A reserved word used by the compiler
to parse programs.  You cannot use keywords, like {\tt public},
{\tt class} and {\tt void} as variable names.

\item[declaration:] A statement that creates a new variable and
determines its type.

\item[assignment:] A statement that assigns a value to a variable.

\item[expression:] A combination of variables, operators and
values that represents a single value.  Expressions also
have types, as determined by their operators and operands.

\item[operator:] A symbol that represents a
computation like addition, multiplication or string
concatenation.

\item[operand:] One of the values on which an operator operates.

\item[precedence:] The order in which operations are evaluated.

\item[concatenate:] To join two operands end-to-end.

\item[composition:] The ability to combine simple
expressions and statements into compound statements and expressions
to represent complex computations concisely.

\index{variable}
\index{value}
\index{type}
\index{keyword}
\index{assignment}
\index{expression}
\index{operator}
\index{concatenate}
\index{operand}
\index{composition}

\end{description}
