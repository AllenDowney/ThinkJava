\section{Vocabulary}

\begin{description}

\term{variable}
A named storage location for values.
All variables have a type, which is declared when the variable is created.

\term{value}
A number or string (or other thing to be named later) that can be stored in a variable.
Every value belongs to a type.

\term{type}
A set of values.
The type of a variable determines which values can be stored there.
The types we have seen are integers ({\tt int} in Java) and strings ({\tt String} in Java).

\term{keyword}
A reserved word used by the compiler to parse programs.
You cannot use keywords, like {\tt public}, {\tt class}, and {\tt void} as variable names.

\term{declaration}
A statement that creates a new variable and determines its type.

\term{assignment}
A statement that assigns a value to a variable.

\term{expression}
A combination of variables, operators and values that represents a single value.
Expressions also have types, as determined by their operators and operands.

\term{operator}
A symbol that represents a computation like addition, multiplication or string concatenation.

\term{operand}
One of the values on which an operator operates.

\term{precedence}
The order in which operations are evaluated.

\term{concatenate}
To join two operands end-to-end.

\term{composition}
The ability to combine simple expressions and statements into compound statements and expressions to represent complex computations concisely.

\end{description}
