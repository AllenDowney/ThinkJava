\section{Exercises}


\begin{exercise}

If you are using this book in a class, you might enjoy this exercise.
Find a partner and play {\it Stump the Chump}:

Start with a program that compiles and runs correctly.
One player turns away while the other player adds an error to the program.
Then the first player tries to find and fix the error.
You get two points if you find the error without compiling the program, one point if you find it using the compiler, and your opponent gets a point if you don't find it.

\end{exercise}


\begin{exercise}
\label{ex:date}

The point of this exercise is (1) to use string concatenation to display values with different types (\java{int} and \java{String}), and (2) to practice developing programs gradually by adding a few statements at a time.

\begin{enumerate}

\item Create a new program named {\tt Date.java}.
Copy or type in something like the ``Hello, World!'' program and make sure you can compile and run it.

\item Following the example in Section~\ref{sec:printvar}, write a program that creates variables named {\tt day}, {\tt date}, {\tt month}, and {\tt year}.
{\tt day} will contain the day of the week and {\tt date} will contain the day of the month.
What type is each variable?
Assign values to those variables that represent today's date.

\item Print the value of each variable on a line by itself.
This is an intermediate step that is useful for checking that everything is working so far.

\item Modify the program so that it prints the date in standard American format, for example: {\tt Thursday, July 16, 2015}.

\item Modify the program again so that the total output is:

\begin{stdout}
American format:
Thursday, July 16, 2015
European format:
Thursday 16 July, 2015
\end{stdout}

\end{enumerate}

HINT: You should be able to copy, paste, and modify the code from Step 4 when completing Step 5.

\end{exercise}


\begin{exercise}

The point of this exercise is (1) to use some of the arithmetic operators, and (2) to start thinking about compound entities (like time of day) that that are represented with multiple values.

\begin{enumerate}

\item Create a new program called {\tt Time.java}.
From now on, we won't remind you to start with a small, working program, but you should.

\item Following the example program in Section~\ref{sec:printvar}, create variables named {\tt hour}, {\tt minute}, and {\tt second}. Assign values that are roughly the current time.
Use a 24-hour clock, i.e., so that at 2pm the value of {\tt hour} is 14.

\item Make the program calculate and print the number of seconds since the most recent midnight.

\item Make the program calculate and print the number of seconds remaining in the day.

\item Make the program calculate and print the percentage of the day that has passed.
You might run into problems when computing percentages with integers, so consider using floating-point.

\item Change the values of {\tt hour}, {\tt minute}, and {\tt second} to reflect the current time.
Check that the program works correctly each time you run it.

\end{enumerate}

HINT: You may want to use additional variables to hold values during the computation.
Variables that are used in a computation but never printed are sometimes called intermediate or temporary variables.

\end{exercise}
