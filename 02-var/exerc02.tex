\section{Exercises}

\begin{exercise}

If you are using this book in a class, you might enjoy this exercise:
find a partner and play "Stump the Chump'':

Start with a program that compiles and runs correctly.  One player
turns away while the other player adds an error to the program.  Then
the first player tries to find and fix the error.  You get two points
if you find the error without compiling the program, one point if you
find it using the compiler, and your opponent gets a point if you
don't find it.

% Note: please don't remove the ``l'' from ``public.''  It's not as
% funny as you think.
\end{exercise}


\begin{exercise}
\label{ex.date}

\begin{enumerate}

\item Create a new program named {\tt Date.java}.  Copy or
type in something like the ``Hello, World'' program and make
sure you can compile and run it.

\item Following the example in Section~\ref{printing}, write a program
that creates variables named {\tt day}, {\tt date}, {\tt month}
and {\tt year}.  {\tt day} will contain the day of the week and {\tt
date} will contain the day of the month.  What type is each variable?
Assign values to those variables that represent today's date.

\item Print the value of each variable on a line by itself.  This is
an intermediate step that is useful for checking that everything is
working so far.

\item Modify the program so that it prints the date in standard
American form: {\tt Saturday, July 16, 2011}.

\item Modify the program again so that the total output is:

\begin{stdout}
American format:
Saturday, July 16, 2011
European format:
Saturday 16 July, 2011
\end{stdout}

\end{enumerate}

The point of this exercise is to use string concatenation to display
values with different types ({\tt int} and {\tt String}), and to
practice developing programs gradually by adding a few statements
at a time.

\end{exercise}


\begin{exercise}

\begin{enumerate}

\item Create a new program called {\tt Time.java}.  From now
on, I won't remind you to start with a small, working program,
but you should.

\item Following the example in Section 2.6, create variables
named {\tt hour}, {\tt minute} and {\tt second}, and assign
them values that are roughly the current time.  Use a 24-hour
clock, so that at 2pm the value of {\tt hour} is 14.

\item Make the program calculate and print the number of
seconds since midnight.

\item Make the program calculate and print the number of
seconds remaining in the day.

\item Make the program calculate and print the percentage of
the day that has passed.

\item Change the values of {\tt hour}, {\tt minute} and {\tt second}
to reflect the current time (I assume that some time has elapsed), and
check to make sure that the program works correctly with different
values.

\end{enumerate}

The point of this exercise is to use some of the arithmetic
operations, and to start thinking about compound entities like the
time of day that that are represented with multiple values.  Also,
you might run into problems computing percentages with {\tt ints},
which is the motivation for floating point numbers in the next
chapter.

HINT: you may want to use additional variables to hold values
temporarily during the computation.  Variables like this, that
are used in a computation but never printed, are sometimes called
intermediate or temporary variables.

\end{exercise}
