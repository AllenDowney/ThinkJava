\section{Exercises}


\begin{exercise}
Draw a stack diagram that shows the state of the program in Section~\ref{recursion}
after {\tt main} invokes {\tt nLines} with the parameter {\tt n=4},
just before the last invocation of {\tt nLines} returns.
\end{exercise}


\begin{exercise}

This exercise reviews the flow of execution through a program
with multiple methods.  Read the following code and answer the
questions below.

\begin{code}
public class Buzz {

    public static void baffle(String blimp) {
        System.out.println(blimp);
        zippo("ping", -5);
    }

    public static void zippo(String quince, int flag) {
        if (flag < 0) {
            System.out.println(quince + " zoop");
        } else {
            System.out.println("ik");
            baffle(quince);
            System.out.println("boo-wa-ha-ha");
        }
    }

    public static void main(String[] args) {
        zippo("rattle", 13);
    }
}
\end{code}

\begin{enumerate}

\item Write the number {\tt 1} next to the first {\em statement}
of this program that will be executed.  Be careful to distinguish
things that are statements from things that are not.

\item Write the number {\tt 2} next to the second statement, and so on
until the end of the program.  If a statement is executed more than
once, it might end up with more than one number next to it.

\item What is the value of the parameter {\tt blimp} when {\tt baffle}
gets invoked?

\item What is the output of this program?

\end{enumerate}
\end{exercise}


\begin{exercise}

The first verse of the song ``99 Bottles of Beer'' is:

\begin{quote}
99 bottles of beer on the wall,
99 bottles of beer,
ya' take one down, ya' pass it around,
98 bottles of beer on the wall.
\end{quote}

Subsequent verses are identical except that the number
of bottles gets smaller by one in each verse, until the
last verse:

\begin{quote}
No bottles of beer on the wall,
no bottles of beer,
ya' can't take one down, ya' can't pass it around,
'cause there are no more bottles of beer on the wall!
\end{quote}

And then the song(finally) ends.

Write a program that prints the entire lyrics of
``99 Bottles of Beer.''  Your program should include a
recursive method that does the hard part, but you
might want to write additional methods to separate the major
functions of the program.

As you develop your code, test it with a small number of
verses, like ``3 Bottles of Beer.''

The purpose of this exercise is to take a problem and break it
into smaller problems, and to solve the smaller problems by writing
simple methods.
\end{exercise}


\begin{exercise}

What is the output of the following program?

\begin{code}
public class Narf {

    public static void zoop(String fred, int bob) {
        System.out.println(fred);
        if (bob == 5) {
            ping("not ");
        } else {
            System.out.println("!");
        }
    }

    public static void main(String[] args) {
        int bizz = 5;
        int buzz = 2;
        zoop("just for", bizz);
        clink(2*buzz);
    }

    public static void clink(int fork) {
        System.out.print("It's ");
        zoop("breakfast ", fork) ;
    }

    public static void ping(String strangStrung) {
        System.out.println("any " + strangStrung + "more ");
    }
}
\end{code}
\end{exercise}


\begin{exercise}
Fermat's Last Theorem says that there are no integers
$a$, $b$, and $c$ such that

\[ a^n + b^n = c^n \]

except in the case when $n=2$.

Write a method named {\tt checkFermat} that takes four
integers as parameters---{\tt a}, {\tt b}, {\tt c} and {\tt n}---and
that checks to see if Fermat's theorem holds.  If
$n$ is greater than 2 and it turns out to be true that
$a^n + b^n = c^n$,
the program should print ``Holy smokes, Fermat was wrong!''
Otherwise the program should print ``No, that doesn't work.''

You should assume that there is a method named {\tt raiseToPow}
that takes two integers as arguments and that raises the
first argument to the power of the second.  For example:

\begin{code}
    int x = raiseToPow(2, 3);
\end{code}

would assign the value {\tt 8} to {\tt x}, because $2^3 = 8$.

\end{exercise}
