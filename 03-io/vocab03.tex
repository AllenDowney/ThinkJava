\section{Vocabulary}

\begin{description}

\term{library}
A collection of packages and classes that are available for use in other programs.
Libraries are often distributed in \java{.jar} (Java Archive) files.

\term{package}
A group of classes that are related to each other.
Java classes are organized into packages.

\term{object}
An abstract entity that represents data and performs actions.
In Java, objects are stored in memory and referenced by variables.

\term{operating system}
Software that is always running behind the scenes on your computer.
It controls the execution of application programs and manages hardware resources.

\term{abstraction}
The process of reducing information and/or detail to focus on high-level concepts.

\term{address}
The storage location of a variable or object in memory.
Addresses are integers encoded in hexadecimal (base 16).

\term{byte}
A single unit of data on a computer; enough to represent one character.

\term{utility class}
A class that provides commonly needed functionality.

%\term{initialization}
%A statement that declares a new variable and assigns a value to it at the same time.
%For example: \java{int sum = 0;}

\term{prototype}
The signature of a method that defines its name and what type it returns.

\term{modulus}
An operator that yields the remainder when one integer is divided by another.
In Java, it is denoted with a percent sign (e.g., \java{5 \% 2} is \java{1}).

\term{floating-point}
A data type that represents a real number (i.e., with multiple decimal places).
In Java, the default floating-point type is \java{double}.

\end{description}
