\section{Vocabulary}

\begin{description}

\term{library}
A collection of packages and classes that are available for use in other programs.
Libraries are often distributed in \java{.jar} (Java Archive) files.

\term{package}
A group of classes that are related to each other.
Java classes are organized into packages.

\term{import}
A statement that allows programs to use classes defined in other packages.

\term{object}
An abstract entity that represents data and performs actions.
In Java, objects are stored in memory and referenced by variables.

\term{operating system}
Software that is always running behind the scenes on your computer.
It controls the execution of application programs and manages hardware resources.

\term{address}
The storage location of a variable or object in memory.
Addresses are integers encoded in hexadecimal (base 16).

\term{abstraction}
The process of reducing information and/or detail to focus on high-level concepts.

\term{byte}
A single unit of data on a computer; enough to represent one character.

\term{utility class}
A class that provides commonly needed functionality.

\term{prototype}
The signature of a method that defines its name and what type it returns.

\term{Javadoc}
A tool that reads Java source code and generates documentation in HTML format.

\term{documentation}
Comments that describe the technical operation of a class or method.

\term{magic number}
A unique value with unexplained meaning or multiple occurrences.
They should generally be replaced with named constants.

\term{format specifier}
A special code beginning with percent sign and ending with a single letter that stands for the data type.

\term{type cast}
An operation that explicitly converts one data type into another, sometimes with loss of information.

\term{truncate}
To make shorter by cutting something off.
Casting a floating-point value to an integer simply removes the fractional part.

\term{modulus}
An operator that yields the remainder when one integer is divided by another.
In Java, it is denoted with a percent sign (e.g., \java{5 \% 2} is \java{1}).

\term{whitespace}
Newlines, tab characters, and other spaces in a source program.
Its purpose in the Java language is to separate tokens.

\term{wildcard}
A command-line feature that allows you to specify a pattern of file names.

\end{description}
