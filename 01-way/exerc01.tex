\section{Exercises}


\begin{exercise}

Computer scientists have the annoying habit of using common English words to mean something other than their common English meaning.
For example, in English, statements and comments are the same thing, but in programs they are different.

The glossary at the end of each chapter is intended to highlight words and phrases that have special meanings in computer science.
When you see familiar words, don't assume that you know what they mean!

\begin{enumerate}
\item In computer jargon, what's the difference between a statement and a comment?
\item What does it mean to say that a program is portable?
\item What is an executable?
\end{enumerate}

\end{exercise}


\begin{exercise}

Before you do anything else, find out how to compile and run a Java program in your environment.
Some environments provide sample programs similar to the example in Section~\ref{sec:hello}.

\begin{enumerate}
\item Type in the ``Hello, world'' program, then compile and run it.

\item Add a print statement that prints a second message after the ``Hello, world!''.
Say something witty like, ``How are you?''
Compile and run the program again.

\item Add a comment to the program (anywhere), recompile, and run it again.
The new comment should not affect the result.
\end{enumerate}

This exercise may seem trivial, but it is the starting place for many of the programs we will work with.
To debug with confidence, you have to have confidence in your programming environment.
In some environments, it is easy to lose track of which program is executing.
You might find yourself trying to debug one program while you are accidentally running another.
Adding (and changing) print statements is a simple way to be sure that the program you are looking at is the program you are running.

\end{exercise}


\begin{exercise}

It is a good idea to commit as many errors as you can think of, so that you see what error messages the compiler produces.
Sometimes the compiler tells you exactly what is wrong, and all you have to do is fix it.
But sometimes the error messages are misleading.
You will develop a sense for when you can trust the compiler and when you have to figure things out yourself.

\begin {enumerate}
\item Remove one of the open squiggly-braces.
\item Remove one of the close squiggly-braces.
\item Instead of {\tt main}, write {\tt mian}.
\item Remove the word {\tt static}.
\item Remove the word {\tt public}.
\item Remove the word {\tt System}.
\item Replace {\tt println} with {\tt Println}.
\item Replace {\tt println} with {\tt print}.
This one is tricky because it is a logic error, not a syntax error.
The statement {\tt System.out.print} is legal, but it may or may not do what you expect.
\item Delete one of the parentheses.  Add an extra one.
\end {enumerate}

\end{exercise}
