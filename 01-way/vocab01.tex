\section{Vocabulary}

\begin{description}

\term{problem-solving}
The process of formulating a problem, finding a solution, and expressing the solution.

\term{program}
A sequence of instructions that specify how to perform tasks on a computer.

\term{programming}
The application of problem-solving to creating executable computer programs.

\term{computer science}
The scientific and practical approach to computation and its applications.

\term{algorithm}
A procedure or formula for solving a problem, with or without a computer.

\term{bug}
An error in a program.

\term{debugging}
The process of finding and removing any of the three kinds of errors.

\term{high-level language}
A programming language that is designed to be easy for humans to read and write.

\term{low-level language}
A programming language that is designed to be easy for a computer to run.
Also called ``machine language'' or ``assembly language.''

\term{portable}
The ability of a program to run on more than one kind of computer.

\term{interpret}
To run a program in a high-level language by translating it one line at a time and immediately executing the corresponding instructions.

\term{compile}
To translate a program in a high-level language into a low-level language, all at once, in preparation for later execution.

\term{source code}
A program in a high-level language, before being compiled.

\term{object code}
The output of the compiler, after translating the program.

\term{executable}
Another name for object code that is ready to run on specific hardware.

\term{byte code}
A special kind of object code used for Java programs.
Byte code is similar to a low-level language, but it is portable like a high-level language.

\term{natural language}
Any of the languages people speak that have evolved naturally.

\term{formal language}
A language people have designed for specific purposes, like representing mathematical ideas or computer programs.

\term{programming language}
A formal language that has been designed to express computations.

\term{syntax}
The structure of a program.

\term{semantics}
The meaning of a program.

\term{token}
A basic element of a program, such as a word, space, symbol, or number.

\term{grammar}
A set of rules that determines whether a statement is legal.

\term{parse}
To examine a program and analyze the syntactic structure.

\term{class}
The blueprints of a program, i.e., what methods it has.

\term{method}
A named sequence of statements.

\term{statement}
A part of a program that specifies a computation.

\term{print statement}
A statement that causes output to be displayed on the screen.

\term{comment}
A part of a program that contains information about the program, but that has no effect when the program runs.

\term{syntax error}
An error in a program that makes it impossible to parse (and therefore impossible to compile).

\term{type-safe}
A property of Java that makes it possible to catch some errors at compile time.

\term{exception}
An error in a program that makes it impossible to execute completely.
Also called a runtime error.

\term{logic error}
An error in a program that makes it do something other than what the programmer intended.

\end{description}
