\begin{quote}
``As we enjoy great Advantages from the Inventions of others,
we should be glad of an Opportunity to serve others by any
Invention of ours, and this we should do freely and generously.''

---Benjamin Franklin, quoted in {\em Benjamin Franklin} by
Edmund S. Morgan.
\end{quote}

\subsection*{Why I wrote this book}

This is the fifth edition of a book I started writing in 1999,
when I was teaching at Colby College.  I had taught an introductory
computer science class using the Java programming language, but I
had not found a textbook I was happy with.  For one thing,
they were all too big!  There was no way my students would read
800 pages of dense, technical material, even if I wanted them to.
And I didn't want them to.  Most of the material was too
specific---details about Java and its libraries that would be obsolete
by the end of the semester, and that obscured the material I really
wanted to get to.

The other problem I found was that the introduction to object-oriented
programming was too abrupt.  Many students who were otherwise
doing well just hit a wall when we got to objects, whether we did
it at the beginning, middle or end.

So I started writing.  I wrote a chapter a day for 13 days, and on
the 14th day I edited.  Then I sent it to be photocopied and bound.
When I handed it out on the first day of class, I told the students
that they would be expected to read one chapter a week.  In other
words, they would read it seven times slower than I wrote it.

\subsection*{The Computer Science AP Exam}

Naturally, when the College Board announced that the AP Exam
would switch to Java, I made plans to update the Java version of
the book.  Looking at the proposed AP Syllabus, I saw that their
subset of Java was all but identical to the subset I had chosen.

During January 2003, I worked on the Fourth Edition of the book,
making these changes:

\begin{itemize}

\item I added sections to improve coverage of the AP syllabus.

\item I improved the appendices on debugging and program development.

\item I collected the exercises, quizzes, and exam questions I
had used in my classes and put them at the end of the appropriate
chapters.  I also made up some problems that are intended to
help with AP Exam preparation.

\end{itemize}

Finally, in August 2011 I wrote the fifth edition, adding
coverage of the GridWorld Case Study that is part of the AP Exam.

\subsection*{Free books!}

Since the beginning, this book has been under a license that allows users
to copy, distribute and modify the book.  Readers can download the
book in a variety of formats and read it on screen or print it.
Teachers are free to print as many copies as they need.  And anyone is
free to customize the book for their needs.

People have translated the book into other computer languages
(including Python and Eiffel), and other natural languages (including
Spanish, French and German).  Many of these derivatives are also
available under free licenses.

Motivated by Open
Source Software, I adopted the philosophy of releasing the
book early and updating it often.  I do my best to minimize the
number of errors, but I also depend on readers to help out.

The response has been great.  I get messages almost every day from
people who have read the book and liked it enough to take the trouble
to send in a ``bug report.''  Often I can correct an error
and post an updated version within a few minutes.  I think of the
book as a work in progress, improving a little whenever I have time
to make a revision, or when readers send feedback.

\subsection*{Oh, the title}

I get a lot of grief about the title of the book.  Not everyone
understands that it is---mostly---a joke.
Reading this book will probably not make you think like a computer
scientist.  That takes time, experience, and probably a few more
classes.

But there is a kernel of truth in the title: this book is not
about Java, and it is only partly about programming.  If it is
successful, this book is about a way of thinking.  Computer scientists
have an approach to problem-solving, and a way of crafting solutions,
that is unique, versatile and powerful.  I hope that this book
gives you a sense of what that approach is, and that at some
point you will find yourself thinking like a computer scientist.

\vspace{0.2in}

\begin{flushleft}
Allen Downey\\
Needham, Massachusetts\\
July 13, 2011
\end{flushleft}
