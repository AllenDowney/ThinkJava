\chapter{Preface}

\begin{quote}
``As we enjoy great Advantages from the Inventions of others,
we should be glad of an Opportunity to serve others by any
Invention of ours, and this we should do freely and generously.''

---Benjamin Franklin, quoted in {\em Benjamin Franklin} by
Edmund S. Morgan.
\end{quote}

\subsection*{Why I wrote this book}

This is the fifth edition of a book I started writing in 1999,
when I was teaching at Colby College.  I had taught an introductory
computer science class using the Java programming language, but I
had not found a textbook I was happy with.  For one thing,
they were all too big!  There was no way my students would read
800 pages of dense, technical material, even if I wanted them to.
And I didn't want them to.  Most of the material was too
specific---details about Java and its libraries that would be obsolete
by the end of the semester, and that obscured the material I really
wanted to get to.

The other problem I found was that the introduction to object-oriented
programming was too abrupt.  Many students who were otherwise
doing well just hit a wall when we got to objects, whether we did
it at the beginning, middle or end.

So I started writing.  I wrote a chapter a day for 13 days, and on
the 14th day I edited.  Then I sent it to be photocopied and bound.
When I handed it out on the first day of class, I told the students
that they would be expected to read one chapter a week.  In other
words, they would read it seven times slower than I wrote it.


\subsection*{The philosophy behind it}

Here are some of the ideas that make the book the way it is:

\begin{itemize}

\item Vocabulary is important.  Students need to be able to talk
about programs and understand what I am saying.  I try to
introduce the minimum number of terms, to define them carefully
when they are first used, and to organize them in glossaries
at the end of each chapter.  In my class, I include vocabulary
questions on quizzes and exams, and require students to use
appropriate terms in short-answer responses.

\item To write a program, students have to understand the
algorithm, know the programming language, and they have to be
able to debug.  I think too many books neglect debugging.  This
book includes an appendix on debugging and an appendix on program
development (which can help avoid debugging).  I recommend that
students read this material early and come back to it often.

\item Some concepts take time to sink in.  Some of the more
difficult ideas in the book, like recursion, appear several times.
By coming back to these ideas, I am trying to give students a
chance to review and reinforce or, if they missed it the first time,
a chance to catch up.

\item I try to use the minimum amount of Java to get the
maximum amount of programming power.  The purpose of this book
is to teach programming and some introductory ideas from computer
science, not Java.  I left out some language features, like
the {\tt switch} statement, that are unnecessary, and avoided
most of the libraries, especially the ones like the AWT that have been
changing quickly or are likely to be replaced.

\end{itemize}

The minimalism of my approach has some advantages.
Each chapter is about ten pages, not including the exercises.
In my classes I ask students to read each chapter before we
discuss it, and I have found that they are willing to do that
and their comprehension is good.  Their preparation makes
class time available for discussion of the more abstract material,
in-class exercises, and additional topics that aren't in the
book.

But minimalism has some disadvantages.  There is not much here
that is intrinsically fun.  Most of my examples demonstrate the
most basic use of a language feature, and many of the exercises
involve string manipulation and mathematical ideas.  I think some
of them are fun, but many of the things that excite students
about computer science, like graphics, sound and network applications,
are given short shrift.

The problem is that many of the more exciting features involve
lots of details and not much concept.  Pedagogically, that means
a lot of effort for not much payoff.  So there is a tradeoff between
the material that students enjoy and the material that is most
intellectually rich.  I leave it to individual teachers to find
the balance that is best for their classes.  To help, the book
includes appendices that cover graphics, keyboard input and
file input.

\subsection*{Object-oriented programming}

Some books introduce objects immediately; others warm up with a more
procedural style and develop object-oriented style more gradually.
This book uses the ``objects late'' approach.

Many of Java's object-oriented features are motivated
by problems with previous languages, and their implementations
are influenced by this history.  Some of these features are
hard to explain if students aren't familiar with the problems
they solve.

It wasn't my intention to postpone object-oriented programming.
On the contrary, I got to it as quickly as I could, limited by
my intention to introduce concepts one at a time, as clearly
as possible, in a way that allows students to practice each
idea in isolation before adding the next.  But I have to admit
that it takes some time to get there.

\subsection*{The Computer Science AP Exam}

Naturally, when the College Board announced that the AP Exam
would switch to Java, I made plans to update the Java version of
the book.  Looking at the proposed AP Syllabus, I saw that their
subset of Java was all but identical to the subset I had chosen.

During January 2003, I worked on the Fourth Edition of the book,
making these changes:

\begin{itemize}

\item I added sections to improve coverage of the AP syllabus.

\item I improved the appendices on debugging and program development.

\item I collected the exercises, quizzes, and exam questions I
had used in my classes and put them at the end of the appropriate
chapters.  I also made up some problems that are intended to
help with AP Exam preparation.

\end{itemize}

Finally, in August 2011 I wrote the fifth edition, adding
coverage of the GridWorld Case Study that is part of the AP Exam.


\subsection*{Free books!}

Since the beginning, this book has been under a license that allows users
to copy, distribute and modify the book.  Readers can download the
book in a variety of formats and read it on screen or print it.
Teachers are free to print as many copies as they need.  And anyone is
free to customize the book for their needs.

People have translated the book into other computer languages
(including Python and Eiffel), and other natural languages (including
Spanish, French and German).  Many of these derivatives are also
available under free licenses.

Motivated by Open
Source Software, I adopted the philosophy of releasing the
book early and updating it often.  I do my best to minimize the
number of errors, but I also depend on readers to help out.

The response has been great.  I get messages almost every day from
people who have read the book and liked it enough to take the trouble
to send in a ``bug report.''  Often I can correct an error
and post an updated version within a few minutes.  I think of the
book as a work in progress, improving a little whenever I have time
to make a revision, or when readers send feedback.

\subsection*{Oh, the title}

I get a lot of grief about the title of the book.  Not everyone
understands that it is---mostly---a joke.
Reading this book will probably not make you think like a computer
scientist.  That takes time, experience, and probably a few more
classes.

But there is a kernel of truth in the title: this book is not
about Java, and it is only partly about programming.  If it is
successful, this book is about a way of thinking.  Computer scientists
have an approach to problem-solving, and a way of crafting solutions,
that is unique, versatile and powerful.  I hope that this book
gives you a sense of what that approach is, and that at some
point you will find yourself thinking like a computer scientist.

\vspace{0.2in}

\begin{flushleft}
Allen Downey\\
Needham, Massachusetts\\
July 13, 2011
\end{flushleft}


\section*{Contributors List}

When I started writing free books, it didn't occur to me to keep
a contributors list.  When Jeff Elkner suggested it, it seemed so
obvious that I am embarassed by the omission.  This list starts
with the 4th Edition, so it omits many people who contributed
suggestions and corrections to earlier versions.

If you have additional comments, please send them to: \\
\href{mailto:feedback@greenteapress.com}{feedback@greenteapress.com}

\begin{itemize}

\item Ellen Hildreth used this book to teach Data Structures at
Wellesley College, and she gave me a whole stack of corrections,
along with some great suggestions.

\item Tania Passfield pointed out that the glossary of Chapter 4
has some leftover terms that no longer appear in the text.

\item Elizabeth Wiethoff noticed that my series expansion of
$\exp(-x^2)$ was wrong.  She is also working on a Ruby version of
the book!

\item Matt Crawford sent in a whole patch file full of corrections!

\item Chi-Yu Li pointed out a typo and an error in one of the code
examples.

\item Doan Thanh Nam corrected an example in Chapter 3.

\item Stijn Debrouwere found a math typo.

\item Muhammad Saied translated the book into Arabic, and found
several errors.

\item Marius Margowski found an inconsistency in a code example.

\item Guy Driesen found several typos.

\item Leslie Klein discovered yet another error in the series expansion
of $\exp(-x^2)$, identified typos in the card array figures, and gave
helpful suggestions to clarify several exercises.

\end{itemize}

Finally, I wish to acknowledge Chris Mayfield for his significant
contribution to version 5.1 of this book. His careful review
lead to over one hundred corrections and improvements throughout.
Several new features include embedded hypertext links and cross
references, consistent layout of all exercises, and Java syntax
highlighting in code examples.

% endcontrib
