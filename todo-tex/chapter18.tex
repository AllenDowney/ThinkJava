\chapter{Input and Output in Java}
\label{javaio}

\section{Keyboard input}
\label{keyboard}
\index{keyboard}
\index{input!keyboard}

First, you have to use {\tt System.in} to create a new
{\tt InputStreamReader}.

\begin{code}
    InputStreamReader in = new InputStreamReader(System.in);
\end{code}

Then you use {\tt in} to create a new {\tt BufferedReader}:

\begin{code}
    BufferedReader keyboard = new BufferedReader(in);
\end{code}

Finally you can invoke {\tt readLine} on {\tt keyboard},
to take input from the keyboard and convert it to a
String.

\begin{code}
    String s = keyboard.readLine();
    System.out.println(s);
\end{code}

There is only one problem.  There are things that can go wrong when
you invoke {\tt readLine}, and they might throw an {\tt IOException}.  A
method that throws an exception has to include it in the
prototype, like this:

\begin{code}
public static void main(String[] args) throws IOException {
    // body of main
}
\end{code}


\section{File input}
\label{fileIO}
\index{file input}
\index{input!file}

Here's a program that reads lines from a file and prints them:

\begin{code}
import java.io.*;

public class Words {

    public static void main(String[] args)
        throws FileNotFoundException, IOException {

        processFile("words.txt");
    }

    public static void processFile(String filename)
        throws FileNotFoundException, IOException {

        FileReader fileReader = new FileReader(filename);
        BufferedReader in = new BufferedReader(fileReader);

        while (true) {
            String s = in.readLine();
            if (s == null) break;
            System.out.println(s);
        }
    }
}
\end{code}

This first line imports {\tt java.io}, the package that contains
{\tt FileReader}, {\tt BufferedReader}, and the rest of
the elaborate class hierarchy Java uses to do
common, simple things.  The {\tt *} means it imports all classes
in the package.
\index{import statement}
\index{statement!import}

Here's what the same program looks like in Python:

\begin{verbatim}
for word in open('words.txt'):
    print word
\end{verbatim}

I'm not kidding.  That's the whole program, and it does the same thing.


\section{Catching exceptions}

In the previous example, {\tt processFile} can throw
{\tt FileNotFoundException} and {\tt IOException}.  And since
{\tt main} calls {\tt processFile}, it has to declare the
same exceptions.  In a larger program, {\tt main} might
declare every exception there is.

The alternative is to {\bf catch} the exception with a
{\tt try} statement.  Here's an example:
\index{try statement}
\index{statement!try}

\begin{code}
    public static void main(String[] args) {
        try {
            processFile("words.txt");
        } catch (Exception ex) {
            System.out.println("That didn't work.  Here's why:");
            ex.printStackTrace();
        }
    }
\end{code}

The structure is similar to an {\tt if} statement.  If the first
``branch'' runs without causing an Exception, the second branch
is skipped.

If the first branch causes an Exception, the flow of execution jumps
to the second branch, which tries to deal with the exceptional
condition (by saying ``error'' in a polite way).  In this case it prints
an error message and the stack trace.

You can download this code from
\url{http://thinkapjava.com/code/Words.java}
and the word list from
\url{http://thinkapjava.com/code/words.txt}.
Make sure both files are in the same folder.
(If you are using an IDE like NetBeans or Eclipse, make
sure the words.txt file is in your project directory.)

Now go do Exercises~\ref{palindrome}, \ref{abecedarian}, and \ref{dupledrome}.



