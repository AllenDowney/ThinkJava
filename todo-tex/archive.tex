\section{Increment and decrement}

\index{operator!increment}
\index{operator!decrement}

Incrementing and decrementing are such common operations that Java provides special operators for them.
The {\tt ++} operator adds one to the current value of an {\tt int} or {\tt char}.
{\tt --} subtracts one.
Neither operator works on {\tt double}s, {\tt boolean}s or {\tt String}s.

Technically, it is legal to increment a variable and use it in an expression at the same time.
For example, you might see something like:

\begin{code}
    System.out.println(i++);
\end{code}

Looking at this, it is not clear whether the increment will take effect before or after the value is printed.
Because expressions like this tend to be confusing, I discourage you from using them.
In fact, to discourage you even more, I'm not going to tell you what the result is.
If you really want to know, you can try it.

Using the increment operators, we can rewrite the letter-counter:

\begin{code}
    int index = 0;
    while (index < length) {
        if (fruit.charAt(index) == 'a') {
            count++;
        }
        index++;
    }
\end{code}

It is a common error to write something like

\begin{code}
    index = index++;             // WRONG!!
\end{code}

Unfortunately, this is syntactically legal, so the compiler will not warn you.
The effect of this statement is to leave the value of {\tt index} unchanged.
This is often a difficult bug to track down.

Remember, you can write {\tt index = index+1}, or you can write {\tt index++}, but you shouldn't mix them.


\section{Objects and primitives}

\index{type!object}
\index{type!primitive}
\index{object type}
\index{primitive type}

There are two kinds of types in Java, primitive types and reference types.
Primitives, like \java{int} and \java{char} begin with lowercase letters; reference types like \java{String} begin with uppercase letters.
This distinction is useful because it reminds us of some of the differences between them:

\begin{itemize}

\item When you declare a primitive variable, you get storage space for a primitive value.
When you declare a reference variable, you only get space for a reference to an object.
To get space for the object itself, you have to use \java{new}.

\item If you don't initialize a primitive type, it is given a default value that depends on the type.
For example, {\tt 0} for {\tt int}s and {\tt false} for {\tt boolean}s.
The default value for object types is {\tt null}, which indicates no object.

\item Primitive variables are well isolated in the sense that there is nothing you can do in one method that will affect a variable in another method.
Object variables can be tricky to work with because they are not as well isolated.
If you pass a reference to an object as an argument, the method you invoke might modify the object, in which case you will see the effect.
Of course, that can be a good thing, but you have to be aware of it.

\end{itemize}

There is one other difference between primitives and object types.
You cannot add new primitives to Java (unless you get yourself on the standards committee), but you can create new object types!  We'll see how in the next chapter.


% was in Chapter 8 Strings and Things
\begin{exercise}
If you did the GridWorld exercises in Chapter~\ref{gridworld}, you
might enjoy this exercise.  The goal is to use trigonometry to get the
Bugs to chase each other.

Make a copy of {\tt BugRunner.java} named {\tt ChaseRunner.java} and
import it into your development environment.  Before you change
anything, check that you can compile and run it.

\begin{itemize}

\item Create two Bugs, one red and one blue.

\item Write a method called {\tt distance} that takes two Bugs
and computes the distance between them.  Remember that you can
get the x-coordinate of a Bug like this:

\begin{code}
    int x = bug.getLocation().getCol();
\end{code}

\item Write a method called {\tt turnToward} that takes two
Bugs and turns one to face the other.  HINT: use {\tt Math.atan2},
but remember that the result is in radians, so you have to
convert to degrees.  Also, for Bugs, 0 degress is North, not East.

\item Write a method called {\tt moveToward} that takes two
Bugs, turns the first to face the second, and then moves the
first one, if it can.

\item Write a method called {\tt moveBugs} that takes two Bugs
and an integer {\tt n}, and moves each Bug toward the other {\tt n}
times.  You can write this method recursively, or use a while loop.

\item Test each of your methods as you develop them.  When they are
  all working, look for opportunities to improve them.  For example,
  if you have redundant code in {\tt distance} and {\tt turnToward},
  you could encapsulate the repeated code in a method.

\end{itemize}
\end{exercise}


% was in Chapter 8 Strings and Things
\begin{exercise}
What is the output of the following program?

\begin{code}
public class Enigma {

    public static void enigma(int x) {
        if (x == 0) {
            return;
        } else {
            enigma(x / 2);
        }
        System.out.print(x % 2);
    }

    public static void main(String[] args) {
        enigma(5);
        System.out.println("");
    }
}
\end{code}

Explain in 4-5 words what the method {\tt enigma} really does.
\end{exercise}


\section{Object methods and class methods}

\index{object method}
\index{method!object}
\index{class method}
\index{method!class}
\index{static}

There are two types of methods in Java, called {\bf class methods} and
{\bf object methods}.  Class methods are identified by the keyword {\tt
  static} in the first line.  Any method that does {\em not} have the
keyword {\tt static} is an object method.

Although we have not written object methods, we have invoked some.
Whenever you invoke a method ``on'' an object, it's an object method.
For example, {\tt charAt} and the other methods we invoked on {\tt String}
objects are all object methods.

For example, here is {\tt printCard} as a class method:

\begin{code}
    public static void printCard(Card c) {
        System.out.println(ranks[c.rank] + " of " + suits[c.suit]);
    }
\end{code}

Anything that can be written as a class method can also be written as an
object method, and vice versa.  But sometimes it is more natural to
use one or the other.
Here it is re-written as an object method:

\begin{code}
    public void print() {
        System.out.println(ranks[rank] + " of " + suits[suit]);
    }
\end{code}

Here are the changes:

\begin{enumerate}

\item I removed {\tt static}.

\item I changed the name of the method to be more idiomatic.

\item I removed the parameter.

\item Inside an object method you can refer to instance variables
as if they were local variables, so I changed {\tt c.rank} to {\tt rank},
and likewise for {\tt suit}.

\end{enumerate}

Here's how this method is invoked:

\begin{code}
    Card card = new Card(1, 1);
    card.print();
\end{code}

When you invoke a method on an object, that object becomes the {\bf
current object}, also known as {\tt this}.  Inside {\tt print},
the keyword {\tt this} refers to the card the method was invoked on.
\index{current object}
\index{object!current}
\index{this}


\section{Oddities and errors}

\index{method!object}
\index{method!class}
\index{overloading}

If you have object methods and class methods in the same class, it is
easy to get confused.  A common way to organize a class definition is
to put all the constructors at the beginning, followed by all the
object methods and then all the class methods.

You can have an object method and a class method with the same
name, as long as they do not have the same number and types of
parameters.  As with other kinds of overloading, Java decides
which version to invoke by looking at the arguments you provide.
\index{static}

Now that we know what the keyword {\tt static} means, you
have probably figured out that {\tt main} is a class method,
which means that there is no ``current object'' when it is invoked.
\index{current object}
\index{this}
\index{instance variable}
\index{variable!instance}
%
Since there is no current object in a class method, it is an
error to use the keyword {\tt this}.  If you try, you get
an error message like: ``Undefined variable: this.''

Also, you cannot refer to instance variables without using dot
notation and providing an object name.  If you try, you get a message
like ``non-static variable... cannot be referenced from a static
context.''  By ``non-static variable'' it means ``instance variable.''


\section{Operations on objects}
\label{objectops}
\index{object}
\index{operator!object}

In the next few
sections, I demonstrate three kinds of methods that
operate on objects:

\begin{description}

\item[pure function:]  Takes objects as
arguments but does not modify them.  The return value is
either a primitive or a new object created inside the method.

\item[modifier:]  Takes objects as arguments and modifies some
or all of them.  Often returns void. \index{void}

\item[fill-in method:]  One of the arguments is an ``empty''
object that gets filled in by the method.  Technically, this is
a type of modifier.

\end{description}

Often it is possible to write a given method as a pure function, a modifier,
or a fill-in method.  I will discuss the pros and cons of each.


\section{Fill-in methods}
\index{fill-in method}
\index{method!fill-in}

Instead of creating a new object every time {\tt addTime} is invoked, we could require the caller to provide an object where {\tt addTime} stores the result.
Compare this to the previous version:

\begin{code}
    public static void addTimeFill(Time t1, Time t2, Time sum) {
        sum.hour = t1.hour + t2.hour;
        sum.minute = t1.minute + t2.minute;
        sum.second = t1.second + t2.second;

        if (sum.second >= 60.0) {
            sum.second -= 60.0;
            sum.minute += 1;
        }
        if (sum.minute >= 60) {
            sum.minute -= 60;
            sum.hour += 1;
        }
    }
\end{code}

The result is stored in {\tt sum}, so the return type is {\tt void}.

Modifiers and fill-in methods are efficient because they don't have to create new objects.
But they make it more difficult to isolate parts of a program; in large projects they can cause errors that are hard to find.

Pure functions help manage the complexity of large projects, in part by making certain kinds of errors impossible.
Also, they lend themselves to certain kinds of composition and nesting.
And because the result of a pure function depends only on the parameters, it is possible to speed them up by storing previously-computed results.

\index{pure function}

I recommend that you write pure functions whenever it is reasonable, and resort to modifiers only if there is a compelling advantage.

%\term{fill-in method}
%A type of method that takes an ``empty'' object as a parameter and fills in its instance variables instead of generating a return value.


\section{Incremental development}

\index{incremental development}
\index{prototyping}
\index{program development!incremental}
\index{program development!planning}

In this chapter I demonstrated a program development process called
{\bf rapid prototyping}\footnote{What I am calling rapid prototyping
  is similar to test-driven development (TDD); the difference is that
  TDD is usually based on automated testing.  See
  \url{http://en.wikipedia.org/wiki/Test-driven_development}.}.  For
each method, I wrote a rough draft that performed the
basic calculation, then tested it on a few cases, correcting flaws
as I found them.

This approach can be effective, but it can lead to code
that is unnecessarily complicated---since it deals with many
special cases---and unreliable---since it is hard to convince
yourself that you have found {\em all} the errors.

An alternative is to look for insight
into the problem that can make the programming easier.  In
this case the insight is that a {\tt Time} is really a three-digit
number in base 60!  The {\tt second} is the ``ones column,''
the {\tt minute} is the ``60's column'', and the {\tt hour}
is the ``3600's column.''

When we wrote {\tt addTime} and {\tt increment}, we were effectively
doing addition in base 60, which is why we had to ``carry'' from one
column to the next.

\index{arithmetic!floating-point}

Another approach to the whole problem is to convert
{\tt Time}s into {\tt double}s and take advantage of the fact that
the computer already knows how to do arithmetic with {\tt double}s.
Here is a method that converts a {\tt Time} into a {\tt double}:

\begin{code}
    public static double convertToSeconds(Time t) {
        int minutes = t.hour * 60 + t.minute;
        double seconds = minutes * 60 + t.second;
        return seconds;
    }
\end{code}

Now all we need is a way to convert from a {\tt double}
to a {\tt Time} object.  We could write a method to
do it, but it might make more sense to write it as a third
constructor:

\begin{code}
    public Time(double secs) {
        this.hour =(int)(secs / 3600.0);
        secs -= this.hour * 3600.0;
        this.minute =(int)(secs / 60.0);
        secs -= this.minute * 60;
        this.second = secs;
    }
\end{code}

This constructor is a little different from the others;
it involves some calculation along with assignments to the
instance variables.

You might have to think to convince yourself that the technique
I am using to convert from one base to another is correct.  But once
you're convinced, we can use these methods to rewrite {\tt addTime}:

\begin{code}
    public static Time addTime(Time t1, Time t2) {
        double seconds = convertToSeconds(t1) + convertToSeconds(t2);
        return new Time(seconds);
    }
\end{code}

This is shorter than the original version, and it is much easier
to demonstrate that it is correct (assuming, as usual, that the
methods it invokes are correct).  As an exercise, rewrite {\tt
increment} the same way.


\section{Generalization and algorithms}

\index{generalization}

In some ways converting from base 60 to base 10 and back is
harder than just dealing with times.  Base conversion is more
abstract; our intuition for dealing with times is better.

But if we have the insight to treat times as base 60 numbers,
and make the investment of writing the conversion methods
({\tt convertToSeconds} and the third constructor), we get
a program that is shorter, easier to read and debug, and more
reliable.

It is also easier to add features later.  Imagine
subtracting two {\tt Time}s to find the duration between them.  The
naive approach would be to implement subtraction complete with
``borrowing.''  Using the conversion methods would be much easier.

Ironically, sometimes making a problem harder (more general)
makes it easier (fewer special cases, fewer opportunities for error).

%TODO show solution?

%\section{Algorithms}
%\label{algorithm}

\index{algorithm}

When you write a general solution for a class of problems, as
opposed to a specific solution to a single problem, you have
written an {\bf algorithm}.  This word is
not easy to define, so I will try a couple of approaches.

First, consider some things that are not algorithms.  When you learned
to multiply single-digit numbers, you probably memorized the
multiplication table.  In effect, you memorized 100 specific
solutions, so that knowledge is not really algorithmic.

But if you were ``lazy,'' you probably learned a few
tricks.  For example, to find the product of $n$ and 9, you can
write $n-1$ as the first digit and $10-n$ as the second digit.  This
trick is a general solution for multiplying any single-digit number by 9.
That's an algorithm!

Similarly, the techniques you learned for addition with carrying,
subtraction with borrowing, and long division are all algorithms.  One
of the characteristics of algorithms is that they do not require any
intelligence to carry out.  They are mechanical processes in which
each step follows from the last according to a simple set of rules.

In my opinion, it is embarrassing that humans spend so much
time in school learning to execute algorithms that,
quite literally, require no intelligence.
%
On the other hand, the process of designing algorithms is
interesting, intellectually challenging, and a central part
of what we call programming.

Some of the things that people do naturally, without difficulty
or conscious thought, are the most difficult to express
algorithmically.  Understanding natural language is a good
example.  We all do it, but so far no one has been able to
explain {\em how} we do it, at least not in the form of an
algorithm.

%~ Soon you will have the opportunity to design
%~ simple algorithms for a variety of problems.

%\item[algorithm:]  A set of instructions for solving a class of
%problems by a mechanical process.


\section{Class definitions and object types}

Here are some syntax issues about class definitions:

\begin{itemize}

\item Class names (and hence object types) should begin with a capital letter, which helps distinguish them from primitive types and variable names.

\item You usually put one class definition in each file, and the name of the file must be the same as the name of the class, with the suffix {\tt .java}.
For example, the \java {class Time} is defined in the file named {\tt Time.java}.

\item In any program, one class is designated as the {\bf startup class} that contains a method named \java{main} where the execution of the program begins.
Other classes {\em may} have a method named \java{main}, but it will not be executed (unless another method calls it directly for some reason).

\end{itemize}

\term{startup class}
The class that contains the \java{main} method where execution of the program begins.

\term{pure function}
A method whose result depends only on its parameters, and that has no side-effects other than returning a value.
