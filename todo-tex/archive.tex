\section{Increment and decrement}

\index{operator!increment}
\index{operator!decrement}

Incrementing and decrementing are such common operations that Java provides special operators for them.
The {\tt ++} operator adds one to the current value of an {\tt int} or {\tt char}.
{\tt --} subtracts one.
Neither operator works on {\tt double}s, {\tt boolean}s or {\tt String}s.

Technically, it is legal to increment a variable and use it in an expression at the same time.
For example, you might see something like:

\begin{code}
    System.out.println(i++);
\end{code}

Looking at this, it is not clear whether the increment will take effect before or after the value is printed.
Because expressions like this tend to be confusing, I discourage you from using them.
In fact, to discourage you even more, I'm not going to tell you what the result is.
If you really want to know, you can try it.

Using the increment operators, we can rewrite the letter-counter:

\begin{code}
    int index = 0;
    while (index < length) {
        if (fruit.charAt(index) == 'a') {
            count++;
        }
        index++;
    }
\end{code}

It is a common error to write something like

\begin{code}
    index = index++;             // WRONG!!
\end{code}

Unfortunately, this is syntactically legal, so the compiler will not warn you.
The effect of this statement is to leave the value of {\tt index} unchanged.
This is often a difficult bug to track down.

Remember, you can write {\tt index = index+1}, or you can write {\tt index++}, but you shouldn't mix them.


\section{Objects and primitives}

\index{type!object}
\index{type!primitive}
\index{object type}
\index{primitive type}

There are two kinds of types in Java, primitive types and reference types.
Primitives, like \java{int} and \java{char} begin with lowercase letters; reference types like \java{String} begin with uppercase letters.
This distinction is useful because it reminds us of some of the differences between them:

\begin{itemize}

\item When you declare a primitive variable, you get storage space for a primitive value.
When you declare a reference variable, you only get space for a reference to an object.
To get space for the object itself, you have to use \java{new}.

\item If you don't initialize a primitive type, it is given a default value that depends on the type.
For example, {\tt 0} for {\tt int}s and {\tt false} for {\tt boolean}s.
The default value for object types is {\tt null}, which indicates no object.

\item Primitive variables are well isolated in the sense that there is nothing you can do in one method that will affect a variable in another method.
Object variables can be tricky to work with because they are not as well isolated.
If you pass a reference to an object as an argument, the method you invoke might modify the object, in which case you will see the effect.
Of course, that can be a good thing, but you have to be aware of it.

\end{itemize}

There is one other difference between primitives and object types.
You cannot add new primitives to Java (unless you get yourself on the standards committee), but you can create new object types!  We'll see how in the next chapter.


% was in Chapter 8 Strings and Things
\begin{exercise}
If you did the GridWorld exercises in Chapter~\ref{gridworld}, you
might enjoy this exercise.  The goal is to use trigonometry to get the
Bugs to chase each other.

Make a copy of {\tt BugRunner.java} named {\tt ChaseRunner.java} and
import it into your development environment.  Before you change
anything, check that you can compile and run it.

\begin{itemize}

\item Create two Bugs, one red and one blue.

\item Write a method called {\tt distance} that takes two Bugs
and computes the distance between them.  Remember that you can
get the x-coordinate of a Bug like this:

\begin{code}
    int x = bug.getLocation().getCol();
\end{code}

\item Write a method called {\tt turnToward} that takes two
Bugs and turns one to face the other.  HINT: use {\tt Math.atan2},
but remember that the result is in radians, so you have to
convert to degrees.  Also, for Bugs, 0 degress is North, not East.

\item Write a method called {\tt moveToward} that takes two
Bugs, turns the first to face the second, and then moves the
first one, if it can.

\item Write a method called {\tt moveBugs} that takes two Bugs
and an integer {\tt n}, and moves each Bug toward the other {\tt n}
times.  You can write this method recursively, or use a while loop.

\item Test each of your methods as you develop them.  When they are
  all working, look for opportunities to improve them.  For example,
  if you have redundant code in {\tt distance} and {\tt turnToward},
  you could encapsulate the repeated code in a method.

\end{itemize}
\end{exercise}


% was in Chapter 8 Strings and Things
\begin{exercise}
What is the output of the following program?

\begin{code}
public class Enigma {

    public static void enigma(int x) {
        if (x == 0) {
            return;
        } else {
            enigma(x / 2);
        }
        System.out.print(x % 2);
    }

    public static void main(String[] args) {
        enigma(5);
        System.out.println("");
    }
}
\end{code}

Explain in 4-5 words what the method {\tt enigma} really does.
\end{exercise}
