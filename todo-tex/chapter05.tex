\chapter{GridWorld: Part 1}
\label{gridworld}

\section{Getting started}

Now is a good time to start working with the AP Computer Science Case
Study, which is a program called GridWorld.  To get started, install
GridWorld, which you can download from the College Board:
\url{http://www.collegeboard.com/student/testing/ap/compsci_a/case.html}.

When you unpack this code, you should have a folder named {\tt
  GridWorldCode} that contains {\tt projects/firstProject}, which
contains {\tt BugRunner.java}.

Make a copy of {\tt BugRunner.java} in another folder and then import it
into your development environment.  There are instructions here that
might help: \url{http://www.collegeboard.com/prod_downloads/student/testing/ap/compsci_a/ap07_gridworld_installation_guide.pdf}.

Once you run {\tt BugRunner.java}, download the GridWorld Student
Manual from \url{http://www.collegeboard.com/prod_downloads/student/testing/ap/compsci_a/ap07_gridworld_studmanual_appends_v3.pdf}.

The Student Manual uses vocabulary I have not presented yet, so to
get you started, here is a quick preview:

\begin{itemize}

\item The components of GridWorld, including Bugs, Rocks and the Grid
itself, are {\bf objects}.

\item A {\bf constructor} is a special method that creates new objects.

\item A {\bf class} is a set of objects; every object belongs to a
class.

\item An object is also called an {\bf instance} because it is a member,
or instance, of a class.

\item An {\bf attribute} is a piece of information about an object, like
its color or location.

\item An {\bf accessor method} is a method that returns an attribute of
an object.

\item A {\bf modifier method} changes an attribute of an object.

\end{itemize}

Now you should be able to read Part 1 of the Student Manual and do
the exercises.

\section{{\tt BugRunner}}

{\tt BugRunner.java} contains this code:

\begin{code}
import info.gridworld.actor.ActorWorld;
import info.gridworld.actor.Bug;
import info.gridworld.actor.Rock;

public class BugRunner
{
    public static void main(String[] args)
    {
        ActorWorld world = new ActorWorld();
        world.add(new Bug());
        world.add(new Rock());
        world.show();
    }
}
\end{code}

The first three lines are {\tt import} statements; they list the
classes from GridWorld used in this program.  You can find the
documentation for these classes at
\url{http://www.greenteapress.com/thinkapjava/javadoc/gridworld/}.

Like the other programs we have seen, BugRunner defines a class
that provides a {\tt main} method.  The first line of {\tt main}
creates an {\tt ActorWorld} object.  {\tt new} is a Java keyword
that creates new objects.

The next two lines create a Bug and a Rock, and add them to {\tt world}.
The last line shows the world on the screen.

Open {\tt BugRunner.java} for editing and replace this line:

\begin{code}
    world.add(new Bug());
\end{code}

with these lines:

\begin{code}
    Bug redBug = new Bug();
    world.add(redBug);
\end{code}

The first line assigns the Bug to a variable named {\tt redBug};
we can use {\tt redBug} to invoke the Bug's methods.  Try this:

\begin{code}
    System.out.println(redBug.getLocation());
\end{code}

Note: If you run this before adding the Bug to the {\tt world}, the result is
{\tt null}, which means that the Bug doesn't have a location yet.

Invoke the other accessor methods and print the bug's attributes.
Invoke the methods {\tt canMove}, {\tt move} and {\tt turn} and
be sure you understand what they do.  % Now try these exercises:


\section{Exercises}

\begin{exercise}

\begin{enumerate}

\item Write a method named {\tt moveBug} that takes a bug as a parameter
and invokes {\tt move}.  Test your method by calling it from {\tt main}.

\item Modify {\tt moveBug} so that it invokes {\tt canMove} and moves
the bug only if it can.

\item Modify {\tt moveBug} so that it takes an integer, {\tt n}, as a
parameter, and moves the bug {\tt n} times (if it can).

\item Modify {\tt moveBug} so that if the bug can't move, it invokes
{\tt turn} instead.

\end{enumerate}
\end{exercise}


\begin{exercise}

\begin{enumerate}

\item The {\tt Math} class provides a method named {\tt random}
that returns a double between 0.0 and 1.0 (not including 1.0).

\item Write a method named {\tt randomBug} that takes a Bug as a
  parameter and sets the Bug's direction to one of 0, 90, 180 or 270
  with equal probability, and then moves the bug if it can.

\item Modify {\tt randomBug} to take an integer {\tt n} and repeat
{\tt n} times.

The result is a random walk, which you can read about
at \url{http://en.wikipedia.org/wiki/Random_walk}.

\item To see a longer random walk, you can give ActorWorld a bigger stage.
At the top of {\tt BugRunner.java}, add this {\tt import} statement:

\begin{code}
import info.gridworld.grid.UnboundedGrid;
\end{code}

Now replace the line that creates the ActorWorld with this:

\begin{code}
    ActorWorld world = new ActorWorld(new UnboundedGrid());
\end{code}

You should be able to run your random walk for a few thousand
steps (you might have to use the scrollbars to find the Bug).

\end{enumerate}
\end{exercise}


\begin{exercise}

GridWorld uses Color objects, which are defined in a Java library.
You can read the documentation at
\url{http://download.oracle.com/javase/6/docs/api/java/awt/Color.html}.

To create Bugs with different colors, we have to import
{\tt Color}:

\begin{code}
import java.awt.Color;
\end{code}

Then you can access the predefined colors, like {\tt Color.blue}, or
create a new color like this:

\begin{code}
    Color purple = new Color(148, 0, 211);
\end{code}

Make a few bugs with different colors.  Then write a method named
{\tt colorBug} that takes a Bug as a parameter, reads its location,
and sets the color.

The Location object you get from
{\tt getLocation} has methods named {\tt getRow} and {\tt getCol} that
return integers.  So you can get the x-coordinate of a Bug like this:

\begin{code}
    int x = bug.getLocation().getCol();
\end{code}

Write a method named {\tt makeBugs} that takes an ActorWorld and an
integer {\tt n} and creates {\tt n} bugs colored according to their
location.  Use the row number to control the red level and the column
to control the blue.

\end{exercise}



