\chapter{Debugging}
\label{debug}
\index{debugging}

The best debugging strategy depends on what kind of error
you have:

\begin{itemize}

\item Syntax errors are produced by the compiler and indicate that
  there is something wrong with the syntax of the program.  Example:
  omitting the semi-colon at the end of a statement.
\index{syntax error}

\item Exceptions are produced if something goes wrong while the
  program is running.  Example: an infinite recursion eventually
  causes a {\tt StackOverflowException}.
\index{exception}
\index{run-time error}

\item Logic errors cause the program to do the wrong thing.  Example:
  an expression may not be evaluated in the order you expect, yielding
  an unexpected result.
\index{logic error}

\end{itemize}

\index{error!syntax}
\index{error!run-time}
\index{error!logic}

The following sections are organized by error type;
some techniques are useful for more than one type.


\section{Syntax errors}

The best kind of debugging is the kind you don't have to do
because you avoid making errors in the first place.  In the
previous section, I suggested development strategies that
minimize errors and makes it easy to find them when you do.
%
The key is to start with a working program and add small
amounts of code at a time.  When there is an error, you will
have a pretty good idea where it is.

Nevertheless, you might find yourself in one of the following
situations.  For each situation, I make some suggestions about
how to proceed.


\subsection*{The compiler is spewing error messages.}
\index{error messages}
\index{compiler}

If the compiler reports 100 error messages, that doesn't mean
there are 100 errors in your program.  When the compiler encounters
an error, it often gets thrown off track for a while.  It tries to
recover and pick up again after the first error, but sometimes
it reports spurious errors.

Only the first error message is truly reliable.  I suggest
that you only fix one error at a time, and then recompile the
program.  You may find that one semi-colon ``fixes'' 100 errors.


\subsection*{I'm getting a weird compiler message and it
won't go away.}

First of all, read the error message carefully.  It is written in
terse jargon, but often there is a carefully hidden kernel of
information.

If nothing else, the message will tell you where in the program the
problem occurred.  Actually, it tells you where the compiler was
when it noticed a problem, which is not necessarily where the error
is.  Use the information the compiler gives you as a guideline,
but if you don't see an error where the compiler is pointing,
broaden the search.

Generally the error will be prior to the location of the error
message, but there are cases where it will be somewhere else
entirely.  For example, if you get an error message at a method
invocation, the actual error may be in the method definition.

If you don't find the error quickly, take a breath and look more
broadly at the entire program.  Make sure the program is indented
properly; that makes it easier to spot syntax errors.

Now, start looking for common errors:
\index{syntax}

\begin{enumerate}

\item Check that all parentheses and brackets are
balanced and properly nested.  All method definitions should be nested
within a class definition.  All program statements should be within a
method definition.

\item Remember that upper case letters are not the same as
lower case letters.

\item Check for semi-colons at the end of statements (and
no semi-colons after squiggly-braces).

\item Make sure that any strings in the code have matching
quotation marks.  Make sure that you use double-quotes for
Strings and single quotes for characters.

\item For each assignment statement, make sure that the type
on the left is the same as the type on the right.  Make sure
that the expression on the left is a variable name or something
else that you can assign a value to (like an element of an array).

\item For each method invocation, make sure that the arguments
you provide are in the right order, and have right type, and that the
object you are invoking the method on is the right type.

\item If you are invoking a value method, make sure you
are doing something with the result.  If you are invoking a
void method, make sure you are {\em not} trying to do something
with the result.

\item If you are invoking an object method, make sure you are
invoking it on an object with the right type.  If you are invoking
a class method from outside the class where it is defined, make
sure you specify the class name.

\item Inside an object method you can refer to the instance
variables without specifying an object.  If you try that in
a class method, you get a message like, ``Static
reference to non-static variable.''

\end{enumerate}

If nothing works, move on to the next section...


\subsection*{I can't get my program to compile no matter
what I do.}

If the compiler says there is an error and you don't see it, that
might be because you and the compiler are not looking at the same
code.  Check your development environment to make sure the program
you are editing is the program the compiler is compiling.  If you
are not sure, try putting an obvious and deliberate syntax error
right at the beginning of the program.  Now compile again.  If
the compiler doesn't find the new error, there is probably something
wrong with the way you set up the development environment.

If you have examined the code thoroughly, and you're sure the compiler
is compiling the right code, it is time for desperate measures:
{\bf debugging by bisection}.
\index{bisection!debugging by}
\index{debugging by bisection}

\begin{itemize}

\item Make a copy of the file you are working on.  If you are
working on {\tt Bob.java}, make a copy called {\tt Bob.java.old}.

\item Delete about half the code from {\tt Bob.java}.  Try compiling
again.

\begin{itemize}

\item If the program compiles now, you know the error is in
the other half.  Bring back about half of the code you deleted and
repeat.

\item If the program still doesn't compile, the error must be in
this half.  Delete about half of the code and repeat.

\end{itemize}

\item Once you have found and fixed the error, start bringing back
the code you deleted, a little bit at a time.

\end{itemize}

This process is ugly, but it goes faster than you might think,
and it is very reliable.


\subsection*{I did what the compiler told me to do, but it
still doesn't work.}

Some compiler messages come with tidbits of advice, like
``class Golfer must be declared
abstract. It does not define int compareTo(java.lang.Object) from
interface java.lang.Comparable.''  It sounds like the compiler
is telling you to declare Golfer as an abstract class, and if
you are reading this book, you probably don't know what that is
or how to do it.

Fortunately, the compiler is wrong.  The solution in this case
is to make sure {\tt Golfer} has a method called {\tt compareTo}
that takes an {\tt Object} as a parameter.

Don't let the compiler lead you by the nose.  Error
messages give you evidence that something is wrong, but
the remedies they suggest are unreliable.


\section{Run-time errors}

\subsection*{My program hangs.}
\index{infinite loop}
\index{infinite recursion}
\index{hanging}

If a program stops and seems to be doing nothing, we
say it is {\bf hanging}.  Often that means that it is caught in
an infinite loop or an infinite recursion.

\begin{itemize}

\item If there is a particular loop that you suspect is the
problem, add a print statement immediately before the loop
that says
``entering the loop'' and another immediately after that
says ``exiting the loop.''

Run the program.  If you get the first message and not
the second, you've got an infinite loop.  Go to the section
titled ``Infinite loop.''

\item Most of the time an infinite recursion will cause the program
to run for a while and then produce a StackOverflowException.
If that happens, go to the section titled ``Infinite recursion.''

If you are not getting a StackOverflowException, but you suspect
there is a problem with a recursive method, you can still use
the techniques in the infinite recursion section.

\item If neither of those suggestions helps, you might not
understand the flow of execution in your program.
Go to the section titled ``Flow of execution.''

\end{itemize}


\subsubsection*{Infinite loop}

If you think you have an infinite loop and you know
which loop it is, add a print statement at
the end of the loop that prints the values of the variables in
the condition, and the value of the condition.

For example,

\begin{code}
    while (x > 0 && y < 0) {
        // do something to x
        // do something to y

        System.out.println("x: " + x);
        System.out.println("y: " + y);
        System.out.println("condition: " + (x > 0 && y < 0));
    }
\end{code}

Now when you run the program you see three lines of output
for each time through the loop.  The last time through the
loop, the condition should be {\tt false}.  If the loop keeps
going, you will see the values of {\tt x} and {\tt y}
and you might figure out why they are not updated correctly.


\subsubsection*{Infinite recursion}

Most of the time an infinite recursion will cause the program
to throw a {\tt StackOverflowException}.  But if the program is
slow it may take a long time to fill the stack.

If you know which method is causing an infinite recursion, check that
there is a base case.  There should be some condition
that makes the method return without making a recursive
invocation.  If not, you need to rethink the algorithm and
identify a base case.

If there is a base case, but the program doesn't seem to be reaching
it, add a print statement at the beginning of the method that prints
the parameters.  Now when you run the program you see a few lines
of output every time the method is invoked, and you see the values of
the parameters.  If the parameters are not moving toward the base case,
you might see why not.


\subsubsection*{Flow of execution}
\index{flow of execution}

If you are not sure how the flow of execution is moving through
your program, add print statements to the beginning of each
method with a message like ``entering method foo,'' where
{\tt foo} is the name of the method.

Now when you run the program it prints a trace of each
method as it is invoked.

You can also print the arguments each method receives.  When you run
the program, check whether the values are reasonable, and check
for one of the most common errors---providing arguments in the wrong
order.


\subsection*{When I run the program I get an Exception.}
\index{Exception}

When an exception occurs, Java prints
a message that includes the name of the
exception, the line of the program where the problem occurred, and a
stack trace.
%
The stack trace includes the method that was running,
the method that invoked it, the method that
invoked {\em that}, and so on.

The first step is to examine the place in the program where
the error occurred and see if you can figure out what happened.

\begin{description}

\item[NullPointerException:] You tried to access an instance
variable or invoke a method on an object that is currently
{\tt null}.  You should figure out which variable is {\tt null}
and then figure out how it got to be that way.

Remember that when you declare a variable with an object type,
it is initially {\tt null} until you assign a value to it.
For example, this code causes a NullPointerException:

\begin{code}
Point blank;
System.out.println(blank.x);
\end{code}

\item[ArrayIndexOutOfBoundsException:] The index you are using
to access an array is either negative or greater than
{\tt array.length-1}.  If you can find the site where the
problem is, add a print statement immediately before it to
print the value of the index and the length of the array.
Is the array the right size?  Is the index the right value?

Now work your way backwards through the program and see where
the array and the index come from.  Find the nearest assignment
statement and see if it is doing the right thing.

If either one is a parameter, go to the place where the method
is invoked and see where the values are coming from.

\item[StackOverFlowException:] See ``Infinite recursion.''

\item[FileNotFoundException:] This means Java didn't find the file
it was looking for.  If you are using a project-based development
environment like Eclipse, you might have to import the file into
the project.  Otherwise make sure the file exists and that the
path is correct.  This problem depends on your file system, so it
can be hard to track down.

\item[ArithmeticException:] Occurs when something goes wrong during
an arithmetic operation, most often division by zero.

\end{description}


\subsection*{I added so many print statements I get inundated with
output.}
\index{print statement}
\index{statement!print}

One of the problems with using print statements for debugging
is that you can end up buried in output.  There are two ways
to proceed: either simplify the output or simplify the program.

To simplify the output, you can remove or comment out print
statements that aren't helping, or combine them, or format
the output so it is easier to understand.  As you develop a program,
you should write code to generate concise,
informative visualizations of what the program is doing.

To simplify the program,
scale down the problem the program is working on.  For example, if you
are sorting an array, sort a {\em small} array.  If the program takes
input from the user, give it the simplest input that causes the
error.

Also, clean up the code.  Remove dead code and reorganize the
program to make it easier to read.  For example, if you
suspect that the error is in a deeply-nested part of the program,
rewrite that part with simpler structure.  If you suspect a
large method, split it into smaller methods and test them
separately.

The process of finding the minimal test case often leads you to the
bug.  For example, if you find that a program works when the array has
an even number of elements, but not when it has an odd number, that
gives you a clue about what is going on.

Reorganizing the program can help you find subtle
bugs.  If you make a change that you think doesn't affect the
program, and it does, that can tip you off.


\section{Logic errors}

\subsection*{My program doesn't work.}

Logic errors are hard to find because the
compiler and the run-time system provide no information about
what is wrong.  Only you know what the program is supposed to
do, and only you know that it isn't doing it.

The first step is to make a connection between the code and the
behavior you get.  You need a hypothesis about what the program
is actually doing.
%
Here are some questions to ask yourself:

\begin{itemize}

\item Is there something the program was supposed to do, but
doesn't seem to be happening?  Find the section of the code
that performs that function and make sure it is executing when
you think it should.  See ``Flow of execution'' above.

\item Is something happening that shouldn't?  Find code in
your program that performs that function and see if it is
executing when it shouldn't.

\item Is a section of code producing an unexpected effect?  Make sure
  you understand the code, especially if it invokes
  Java methods.  Read the documentation for those methods, and
  try them out with simple test cases.  They might not do what you
  think they do.

\end{itemize}

To program, you need a mental model what your code does.
If it doesn't do what you expect, the problem might not be
the program; it might be in your head.
\index{model!mental}
\index{mental model}

The best way to correct your mental model is to break the program
into components (usually the classes and methods) and test
them independently.  Once you find the discrepancy
between your model and reality, you can solve the problem.

Here are some common logic errors to check for:

\begin{itemize}

\item Remember that integer division always rounds down.  If you
want fractions, use {\tt doubles}.

\item Floating-point numbers are only approximate, so don't rely
on perfect accuracy.

\item More generally, use integers for countable things
and floating-point numbers for measurable things.

\item If you use the assignment operator ({\tt =})
instead of the equality operator ({\tt ==}) in the
condition of an {\tt if}, {\tt while}, or {\tt for} statement,
you might get an expression that is syntactically legal and
semantically wrong.

\item When you apply the equality operator ({\tt ==}) to an
object, it checks identity.  If you meant to check
equivalence, you should use the {\tt equals} method.

\item For user defined types, {\tt equals} checks identity.
If you want a different notion of equivalence, you have to
override it.

\item Inheritance can lead to subtle logic errors,
because you can run inherited code without realizing it.
See ``Flow of Execution'' above.

\end{itemize}


\subsection*{I've got a big hairy expression and it doesn't
do what I expect.}
\index{expression!big and hairy}

Writing complex expressions is fine as long as they are readable,
but they can be hard to debug.  It is often a good idea to
break a complex expression into a series of assignments to
temporary variables.

For example:

\begin{code}
rect.setLocation(rect.getLocation().translate(
                -rect.getWidth(), -rect.getHeight()));
\end{code}

Can be rewritten as

\begin{code}
int dx = -rect.getWidth();
int dy = -rect.getHeight();
Point location = rect.getLocation();
Point newLocation = location.translate(dx, dy);
rect.setLocation(newLocation);
\end{code}

The explicit version is easier to read, because the variable
names provide additional documentation, and easier to debug,
because you can check the types of the temporary variables
and display their values.

\index{temporary variable}
\index{variable!temporary}
\index{order of evaluation}
\index{precedence}

Another problem that can occur with big expressions is
that the order of evaluation may not be what you expect.
For example, to evaluate
$\frac{x}{2 \pi}$, you might write

\begin{code}
double y = x / 2 * Math.PI;
\end{code}

That is not correct, because multiplication and division have
the same precedence, and they are evaluated from left to right.
This expression computes $x \pi / 2$.

If you are not sure of the order of operations, use parentheses to
make it explicit.

\begin{code}
double y = x / (2 * Math.PI);
\end{code}

This version is correct,
and more readable for
other people who haven't memorized the order of operations.



\subsection*{My method doesn't return what I expect.}
\index{return statement}
\index{statement!return}

If you have a return statement with a complex expression,
you don't have a chance to print the value before
returning.  Again, you can use a temporary variable.  For
example, instead of

\begin{code}
public Rectangle intersection(Rectangle a, Rectangle b) {
    return new Rectangle(
        Math.min(a.x, b.x),
        Math.min(a.y, b.y),
        Math.max(a.x+a.width, b.x+b.width)-Math.min(a.x, b.x)
        Math.max(a.y+a.height, b.y+b.height)-Math.min(a.y, b.y) );
}
\end{code}

You could write

\begin{code}
public Rectangle intersection(Rectangle a, Rectangle b) {
    int x1 = Math.min(a.x, b.x);
    int y2 = Math.min(a.y, b.y);
    int x2 = Math.max(a.x+a.width, b.x+b.width);
    int y2 = Math.max(a.y+a.height, b.y+b.height);
    Rectangle rect = new Rectangle(x1, y1, x2-x1, y2-y1);
    return rect;
}
\end{code}

Now you have the opportunity to display any of
the intermediate variables before returning.  And by
reusing {\tt x1} and {\tt y1}, you made the code smaller, too.


\subsection*{My print statement isn't doing anything}
\index{print statement}
\index{statement!print}

If you use the {\tt println} method, the output is displayed
immediately, but if you use {\tt print} (at least in some
environments) the output gets stored without being displayed until the
next newline.  If the program terminates without
printing a newline, you may never see the stored output.

If you suspect that this is happening to, change
some or all of the {\tt print} statements to {\tt println}.


\subsection*{I'm really, really stuck and I need help}

First, get away from the computer for a few minutes.
Computers emit waves that affect the brain, causing the following
symptoms:

\begin{itemize}

\item Frustration and rage.

\item Superstitious beliefs (``the computer hates me'') and
magical thinking (``the program only works when I wear my
hat backwards'').

\item Sour grapes (``this program is lame anyway'').

\end{itemize}

If you suffer from any of these symptoms, get
up and go for a walk.  When you are calm, think about the program.
What is it doing?  What are possible causes of that
behavior?  When was the last time you had a working program,
and what did you do next?

Sometimes it just takes time to find a bug.  I often find bugs when I
let my mind wander.  Good places to find bugs are trains, showers, and
bed.


\subsection*{No, I really need help.}

It happens.  Even the best programmers get stuck.
Sometimes you need a fresh pair of eyes.

Before you bring someone else in, make sure you have tried
the techniques described above.  Your program should be as simple
as possible, and you should be working on the smallest input
that causes the error.  You should have print statements in the
appropriate places (and the output they produce should be
comprehensible).  You should understand the problem well enough
to describe it concisely.

When you bring someone in to help, give
them the information they need.

\begin{itemize}

\item What kind of bug is it?  Syntax, run-time, or logic?

\item What was the last thing you did before this error occurred?
What were the last lines of code that you wrote, or what is
the new test case that fails?

\item If the bug occurs at compile-time or run-time, what is
the error message, and what part of the program does it indicate?

\item What have you tried, and what have you learned?

\end{itemize}

By the time you explain the problem
to someone, you might see the answer.  This phenomenon
is so common that some people recommend a debugging technique
called ``rubber ducking.''  Here's how it works:

\begin{enumerate}

\item Buy a standard-issue rubber duck.

\item When you are really stuck on a problem, put the rubber
duck on the desk in front of you and say, ``Rubber duck, I
am stuck on a problem.  Here's what's happening...''

\item Explain the problem to the rubber duck.

\item See the solution.

\item Thank the rubber duck.

\end{enumerate}

I am not kidding.  See
\url{http://en.wikipedia.org/wiki/Rubber_duck_debugging}.


\subsection*{I found the bug!}

When you find the bug, it is usually obvious how to fix it.  But not
always.  Sometimes what seems to be a bug is really an indication that
you don't understand the program, or there is an error in your
algorithm.  In these cases, you might have to rethink the algorithm,
or adjust your mental model.  Take some time away from
the computer to think, work through test cases
by hand, or draw diagrams to represent the computation.

After you fix the bug, don't just start in making new errors.
Take a minute to think about what kind of bug it was, why you made
the error, how the error manifested itself, and
what you could have done to find it faster.  Next time you see something
similar, you will be able to find the bug more quickly.


