\section{Exercises}


\begin{exercise}

What is the difference between a variable and a method?
In terms of their syntax, how does the Java compiler tell the difference between the two?

%A variable is a {\em location of data}, whereas a method is a {\em location of code}.
%In Java, methods always have parentheses, even if they have no arguments like \java{System.out.println()}.

\end{exercise}


\begin{exercise}

The point of this exercise is to practice reading code and to make sure that you understand the flow of execution through a program with multiple methods.

\begin{enumerate}

\item What is the output of the following program?
Be precise about where there are spaces and where there are newlines.

HINT: Start by describing in words what {\tt ping} and {\tt baffle} do when they are invoked.

\item Draw a stack diagram that shows the state of the program the first time {\tt ping} is invoked.

\end{enumerate}

\begin{code}
    public static void zoop() {
        baffle();
        System.out.print("You wugga ");
        baffle();
    }

    public static void main(String[] args) {
        System.out.print("No, I ");
        zoop();
        System.out.print("I ");
        baffle();
    }

    public static void baffle() {
        System.out.print("wug");
        ping();
    }

    public static void ping() {
        System.out.println(".");
    }
\end{code}

\end{exercise}


\begin{exercise}

Draw a stack diagram that shows the state of the program in Section~\ref{time} when {\tt main} invokes {\tt printTime} with the arguments {\tt 11} and {\tt 59}.

\end{exercise}


\begin{exercise}

The point of this exercise is to make sure you understand how to write and invoke methods that take parameters.

\begin{enumerate}
\item Write the first line of a method named {\tt zool} that takes three parameters: an {\tt int} and two {\tt Strings}.

\item Write a line of code that invokes {\tt zool}, passing as arguments the value {\tt 11}, the name of your first pet, and the name of the street you grew up on.
\end{enumerate}

\end{exercise}


\begin{exercise}

The purpose of this exercise is to take code from a previous exercise and encapsulate it in a method that takes parameters.
You should start with a working solution to Exercise~\ref{ex:date}.

\begin{enumerate}

\item Write a method called {\tt printAmerican} that takes the day, date, month and year as parameters and that prints them in American format.

\item Test your method by invoking it from {\tt main} and passing appropriate arguments.
The output should look something like this (except that the date might be different):

\begin{stdout}
Saturday, July 16, 2011
\end{stdout}

\item Once you have debugged {\tt printAmerican}, write another method called {\tt printEuropean} that prints the date in European format.

\end{enumerate}
\end{exercise}
