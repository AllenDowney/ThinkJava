\section{Vocabulary}

\begin{description}

% Note: expanded definition from Chapter 1
\term{method}
A named sequence of statements that performs a procedure or function.
Methods may or may not take parameters, and may or may not return a value.

\term{invoke}
To call a method, i.e., cause it to execute.

\term{parameter}
A piece of information a method requires before it can run.
Parameters are variables: they contain values and have types.

\term{argument}
A value that you provide when you invoke a method.
This value must have the same type as the corresponding parameter.

% Note: expanded definition from Chapter 2
\term{composition}
The ability to combine simple expressions and statements into compound expressions and statements, making it possible to use intermediate computations as arguments.

\term{order of execution}
The order in which Java executes methods and statements.
It may not necessarily be from top to bottom, left to right.

\term{parameter passing}
The process of assigning an argument value to a parameter variable.

\term{stack diagram}
A memory diagram that shows which variables belong to each method at a certain point in the program.
The methods are ``stacked'' from top to bottom in the order of execution.

\term{frame}
A structure (represented by a box in stack diagrams) that contains a method's parameters and variables.

\term{scope}
The area of a program where a variable exists.

\term{debugger}
A tool that allows you to run one statement at a time and see the contents of variables.

\term{breakpoint}
A line of code where the debugger will pause a running program.

\term{call stack}
The history of method calls and where to resume execution after each method returns.

\term{rounding error}
The small difference between a floating-point number and its actual representation on computer hardware.

\end{description}
