\section{Vocabulary}

\begin{description}

\item[class:]  A named collection of methods.  So far, we have used
the {\tt Math} class and the {\tt System} class, and we have
written classes named {\tt Hello} and {\tt NewLine}.

\item[method:]  A named sequence of statements that performs a
useful function.  Methods may or may not take parameters, and may
or may not return a value.

\item[parameter:]  A piece of information a method requires before
it can run.  Parameters are variables: they contain values and have types.

\item[argument:]  A value that you provide when you invoke a
method.  This value must have the same type as the corresponding
parameter.

\item[frame:] A structure (represented by a gray box in stack diagrams)
that contains a method's parameters and variables.

\item[invoke:]  Cause a method to execute.

\end{description}
